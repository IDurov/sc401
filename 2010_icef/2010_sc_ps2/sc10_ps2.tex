\documentclass[pdftex,12pt,a4paper]{article}

\input{/home/boris/Dropbox/Public/tex_general/title_bor_utf8}

%\usepackage{showkeys} % показывать метки

\input{/home/boris/Dropbox/Public/tex_general/prob_and_sol_utf8}

%\title{Задачи по элементарной теории вероятностей и матстатистике}
%\author{Составитель: Борис Демешев, boris.demeshev@gmail.com}
%\date{\today}

\begin{document}

%\pagestyle{myheadings} \markboth{ТВИМС-задачник. Демешев Борис. roah@yandex.ru }{ТВИМС-задачник. Демешев Борис. roah@yandex.ru }
%\maketitle
%\tableofcontents{}

%\parindent=0 pt % отступ равен 0




Задачи для подготовки к контрольной. Сдавать ни одну задачу не нужно.


\problemonly

\problem{ Пусть $X$ и $Y$ - независимые биномиально распределенные случайные величины, $X$ - $Bin(n_{1},p)$, $Y$ - $Bin(n_{2},p)$. Найдите $E(X|X+Y)$.}
\solution{$E(X|X+Y)=\frac{n_{1}}{n_{1}+n_{2}}(X+Y)$}

%\problem{ Пусть $X$ и $Y$ - независимые пуассоновские случайные величины с параметрами $\lambda_{1}$ и $\lambda_{2}$. Найдите $E(X|X+Y)$.}
%\solution{$E(X|X+Y)=\frac{\lambda_{1}}{\lambda_{1}+\lambda_{2}}(X+Y)$}


\problem{Введем понятие условной дисперсии. По определению, $ Var(X|\mathcal{H}):=E(X^{2}|\mathcal{H})-(E(X|\mathcal{H}))^{2}$. Докажите, что $ Var(X)=E(Var(X|\mathcal{H}))+Var(E(X|\mathcal{H})) $}
\solution{}

\problem{Пусть $ X $ и $ Y $ имеют совместное нормальное распределение, $ E(X)=E(Y)=0 $, $ Var(X)=Var(Y)=1 $ и $ Corr(X,Y)=\rho $. Найдите $ E(X|Y) $, $ E(Y|X) $, $ Var(X|Y) $, $ Var(Y|X) $}
\solution{}



\problem{
Монетка выпадает орлом с вероятностью $p$. Эксперимент состоит из двух этапов. На первом этапе монетку подкидывают 100 раз и записывают число орлов. На втором этапе монетку подбрасывают до тех пор пока не выпадет столько орлов, сколько выпало на первом этапе. Обозначим число подбрасываний монетки на втором этапе буквой $X$. 

Найдите $E(X)$, $Var(X)$.

Hint: $ E(X)=E(E(X|Y)) $, $ Var(X)=E(Var(X|Y))+Var(E(X|Y)) $ }
\solution{}



\problem{ Вася случайно выбирает между 0 и 1 число $X_1$, затем случайно выбирает между 0 и $X_1$ число $X_2$, затем $X_3$ между 0 и $X_{2}$, и так до бесконечности.
\begin{enumerate}
  \item Найдите $E(X_n)$, $Var(X_n)$;
  \item Найдите функцию плотности распределения $X_n$;
  \item Найдите $E(X_2|X_1,X_3)$;
\item К какой случайной величине стремится $X_n$ и в каких смыслах?
\end{enumerate}
}
\solution{
$E(X_n)=\frac{1}{2^n}$; $V(X_n)=\frac{1}{3^n}-\frac{1}{4^n}$. $f_{X_n}(x)=\frac{(-\ln{x})^{n-1}}{(n-1)!}\cdot 1_{[0;1]}.$\\
$E(X_2|X_1,X_3)=\frac{X_1-X_3}{\ln{X_1}-\ln{X_3}}.$\\
При любом $\omega$ последовательность $X_n(\omega)$ монотонно невозрастающая и ограниченная (нулем). Значит, она сходится. При этом
 $$P(\forall i \mbox{ } X_i>\epsilon)<P(X_k>\epsilon)<\frac{E(X_k)}{\epsilon}=\frac{1}{2^k\epsilon}$$ для любого $k$. Значит, $P(\forall i \mbox{ }X_i>\epsilon)$ меньше любого положительного числа, то есть $P(\forall i\mbox{ }X_i>\epsilon)=0$. Значит, с вероятностью 1 $X_n(\omega)$ сходится именно к нулю, то есть $X_n\to 0$ почти наверное. Отсюда следуют сходимости по вероятности и по распределению. Поскольку $\lim\limits_{n\to \infty}E(X_n)=0$, а $0<E(X_n^p)<E(X_n)$ для $\forall p>1$, то имеет место также и сходимость в $L^p$. Итак, данная последовательность стремится к нулю во всех разумных смыслах.
 
Источник: Алексей Суздальцев}


\problem{Приведите пример последовательности $X_{n}$ и предела $X$, таких что: $\lim E|X_{n}-X|=0$, но $E(|X_{n}|)=\infty$ }
\solution{ В качестве $X_{n}$ и $ X $ берем любую случайную величину с бесконечным мат. ожиданием}


\problem{Пусть $ X\sim U[0;1] $. К чему и в каких смыслах сходится $ X_{n}=1_{X>\frac{n}{2n+1}} $}
\solution{$ 1_{X>0.5} $}




\problem{Пусть $ X_{i} $ независимы и одинаково распределены и $ E(X_{i})=2010 $. К чему и в каких смыслах сходится $ \bar{X}_{n}:=\frac{X_{1}+...+X_{n}}{n} $? (доказывать закон больших чисел не нужно, следует просто вспомнить его формулировку)}
\solution{}

\problem{На столе хорошо перемешанная колода из 36 карт. Карты открывают одну за одной до появления первого туза. Какова вероятность того, что следующая карта будет тузом?

Hint: а может быть доля тузов в неоткрытой части колоды --- это мартингал? Впрочем, можно и без мартингалов обойтись. }
\solution{}

\problem{Изначально воюют один воин Добра и один воин Зла. Каждый день судьба выбирает одного из воюющих наугад и добавляет еще одного воина той же стороны. Никто никогда не погибает, они просто сражаются. Пусть $T$ - время, когда Судьба впервые добавит война Добра.  Найдите $E(1/(T+2)) $.

Hint: может доля войнов Добра в конце дня $ t $ --- мартингал?

}
\solution{}

\end{document}


%\bibliography{/home/boris/Dropbox/Public/tex_general/opit}
%\printindex % печать предметного указателя здесь

