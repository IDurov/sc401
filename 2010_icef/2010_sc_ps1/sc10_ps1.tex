\documentclass[pdftex,12pt,a4paper]{article}

\input{/home/boris/Dropbox/Public/tex_general/title_bor_utf8}

%\usepackage{showkeys} % показывать метки

\input{/home/boris/Dropbox/Public/tex_general/prob_and_sol_utf8}

%\title{Задачи по элементарной теории вероятностей и матстатистике}
%\author{Составитель: Борис Демешев, boris.demeshev@gmail.com}
%\date{\today}

\begin{document}

%\pagestyle{myheadings} \markboth{ТВИМС-задачник. Демешев Борис. roah@yandex.ru }{ТВИМС-задачник. Демешев Борис. roah@yandex.ru }
%\maketitle
%\tableofcontents{}

%\parindent=0 pt % отступ равен 0




Задачи для подготовки к контрольной. Сдавать ни одну задачу не нужно.


\problemonly


\problem{ Пусть $A$ --- множество всех подмножеств натуральных чисел, а $ B $ --- множество бесконечных последовательностей из 0 и 1. Для примера: $\{5,6,178\}\in A$, $01010101010101.....\in B$. Сравните мощность множеств $ A $ и $ B $.}
\solution{}


\problem{Монетка подкидывается бесконечное количество раз: $X_{n}$ равно 1, если при $ n $-ом подбрасывании выпадает орел и 0, если решка. Определим кучу $ \sigma $-алгебр: $\mathcal{F}_{n}:=\sigma(X_{1},X_{2},...,X_{n})$, $\mathcal{H}_{n}:=\sigma(X_{n},X_{n+1},X_{n+2},...)$.

Приведите по два нетривиальных (т.е. $ \Omega $ и $ \emptyset $ не называть) примера такого события $ A $, что:

\begin{itemize}
\item $ A\in \mathcal{F}_{2010} $
\item $ A\notin \mathcal{F}_{2010} $
\item $A$ лежит в каждой $\mathcal{H}_{n}$
\end{itemize}

В какие из упомянутых $ \sigma $-алгебр входят события:
\begin{itemize}
\item $ X_{37}>0$
\item $ X_{37}>X_{2010}$
\item $ X_{37}>X_{2010}>X_{12}$
\end{itemize}

Упростите выражения: $ \mathcal{F}_{11}\cap \mathcal{F}_{25} $, $ \mathcal{F}_{11}\cup \mathcal{F}_{25} $, $ \mathcal{H}_{11}\cap \mathcal{H}_{25} $, $ \mathcal{H}_{11}\cup \mathcal{H}_{25} $}
\solution{}


\problem{Может ли в $ \sigma $-алгебре быть ровно 2010 элементов?}
\solution{Нет. В конечной только $ 2^{n} $, где $ n $ - количество элементов в разбиении $ \Omega $}


\problem{Пусть $X$ - равномерная на $ [0;1] $ случайная величина. Пусть $ \mathcal{H}_{1} $ --- минимальная $ \sigma $-алгебра, содержащая все события вида $ X=t $, а $ \mathcal{H}_{2} $ --- минимальная $ \sigma $-алгебра, содержащая все события вида $X<t$. Сравните $\sigma$-алгебры $ \mathcal{H}_{1} $ и $ \mathcal{H}_{2} $.}
\solution{$ \mathcal{H}_{1} \subset \mathcal{H}_{2} $}


\problem{Правильная монетка подбрасывается бесконечное количество раз. Вася наблюдает за результатами подбрасываний до тех пор, пока не выпадет 3 орла подряд. Пусть $ T $ - случайный момент времени, когда Вася прекратил наблюдения, и $ \mathcal{F}_{T} $ --- $ \sigma $-алгебра событий различимых Васей. Приведите пример двух нетривиальных (т.е. не $ \Omega $ и не $ \emptyset $) событий входящих в $ \mathcal{F}_{T} $.}
\solution{}


\problem{ Правильный кубик подбрасывается один раз. $X$ - число очков, выпавшее на кубике. $Y$ - индикатор того, выпала ли четная грань. $Z$ - индикатор того, выпало ли число больше 2-х. \\
Найдите закон распределения (проще говоря, заполните табличку) для случайных величин $E(XY|XZ)$, $E(Z|X)$, $E(X|Z)$ \\
Табличка для заполнения: \\
$\begin{array}{|c|c|c|c|c|c|c|} 
\hline
\Omega & w_{1} & w_{2} & w_{3} & w_{4} & w_{5} & w_{6} \\
\hline
E(XY|XZ) & & & & & & \\
\hline
E(Z|X) & & & & & & \\
\hline
E(X|Z) & & & & & & \\
\hline
\end{array}$}
\solution{}



\problem{ Пусть совместное распределение $X$ и $Y$ задано таблицей: \\
\begin{tabular}{|c|c|c|}
  \hline
  % after \\: \hline or \cline{col1-col2} \cline{col3-col4} ...
   & $X=-1$ & $X=1$ \\
  \hline
  $Y=-1$ & 1/8 & 4/8 \\
  $Y=2$ & 2/8 & 1/8 \\
  \hline
\end{tabular} \\
а) Найдите $E(X|Y)$, представьте ответ в виде $E(X|Y)=a+bY$. \\
б) Убедитесь, что $E(E(X|Y))=E(X)$ \\
в) Найдите $E(XY|Y)$ и представьте ответ в виде $f(Y)$ }
\solution{}

\problem{
Find conditional expectation $E(X|Y)$ intuitively (without formal proof)  \\
$Z\sim U[0;1]$, $A=\{Z>0.5\}$, $X=2Z^{2}$, $Y=2Z-1_{A}$.}
\solution{}


\problem{
Маша собрала $n$ грибов в лесу наугад. Рыжики попадаются с вероятностью $r$, лисички - с вероятностью $l$, где $r+l<1$. Пусть $R$ - количество собранных рыжиков, $L$ - лисичек. Найдите:
\begin{enumerate}
\item $E(R|L)$
%\item $Var(R|L)$
\item $P(E(R|L)=0)$
\end{enumerate}}
\solution{$E(R|L)=(n-L)\frac{r}{1-l}$, $Var(R|L)=(n-L)\frac{r}{1-l}\frac{1-r-l}{1-l}$, $P(E(R|L)=0)=l^{n}$}


\problem{
Случайные величины $X$ и $Y$ независимы и одинаково распределены.
Найдите $E(X|X+Y)$, $E(X-Y|X+Y)$, $E(X^{2}-Y^{2}|X+Y)$ }
\solution{
$E(X|X+Y)=0.5(X+Y)$, $E(X-Y|X+Y)=0$, $E(X^{2}-Y^{2}|X+Y)=0$ }


%\bibliography{/home/boris/Dropbox/Public/tex_general/opit}
%\printindex % печать предметного указателя здесь

\end{document}