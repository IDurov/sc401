\documentclass[pdftex,12pt,a4paper]{article}

\input{/home/boris/science/tex_general/title_bor_utf8}

%\usepackage{showkeys} % показывать метки

\input{/home/boris/science/tex_general/prob_and_sol_utf8}

\title{Разные тексты по теории вероятностей}
\author{Борис Демешев, \url{boris.demeshev@gmail.com}}
\date{\today}


\newcommand{\indef}[1]{\textbf{#1}}


\numberwithin{equation}{page} % уравнения нумеруются на каждой стр. отдельно


\newtheorem{myth}[equation]{Теорема} % нумерация сквозная с уравнениями

\theoremstyle{definition} % убирает курсив и что-то еще наверное делает ;)
\newtheorem{mydef}[equation]{Определение}

\theoremstyle{definition}
\newtheorem{myex}[equation]{Пример}

%\newtheorem{assertion}{Утверждение}
%\newtheorem{lemma}{Лемма}

\theoremstyle{definition}
\newtheorem*{myproof}{Доказательство}

%\newtheorem{problem}{Задача}

\makeindex % команда для создания предметного указателя


\bibliographystyle{plain} % стиль оформления ссылок



\begin{document}

%\maketitle
%\tableofcontents{}


%\pagestyle{myheadings} \markboth{ТВИМС-задачник. Демешев Борис. roah@yandex.ru }{ТВИМС-задачник. Демешев Борис. roah@yandex.ru }
%\maketitle
%\tableofcontents{}

%\parindent=0 pt % отступ равен 0

%\part{Многомерное нормальное распределение}
%\subsubsection*{Векторные случайные величины}

Векторная случайная величина - это просто вектор-столбец из случайных величин:

\begin{equation}
\vec{X}=\left(\begin{array}{c} X_{1} \\ X_{2} \\ ... \\ X_{n} \\  \end{array} \right)
\end{equation}

В принципе можно говорить о векторе-строке или о случайной матрице, но нам понадобятся только вектор-столбцы. Две самых распространенных характеристики одномерной случайной величины $X$ - это среднее $\E(X)$ и дисперсия $Var(X)$.
У векторных случайных величин также есть среднее:

\begin{equation}
\E(\vec{X})=\left(\begin{array}{c} \E(X_{1}) \\ \E(X_{2}) \\ ... \\ \E(X_{n}) \\  \end{array} \right)
\end{equation}



И ковариационная матрица:

\begin{equation}
Var(\vec{X})=\left(
\begin{array}{cccc} 
Var(X_{1}) & Cov(X_{1},X_{2}) & ... & Cov(X_{1},X_{n}) \\ 
Cov(X_{2},X_{1}) & Var(X_{2}) & ... & Cov(X_{2},X_{n}) \\ 
... &&&\\ 
Cov(X_{n},X_{1}) & Cov(X_{n},X_{1})  & ... & Var(X_{n})\\ 
\end{array} 
\right)
\end{equation}

В ковариационной матрицы в $i$-ой строке в $j$-ом столбце находится $Cov(X_{i},X_{j})$. На диагонали $i$-ый элемент, это $Cov(X_{i},X_{i})=Var(X_{i})$. Из-за этого сразу следует, что ковариационная матрица симметрична.

Для простоты мы будем обозначать транспонирование с помощью штриха, вот так: $A'$. Путаницы с производной не возникнет.

С помощью транспонирования легко сказать, что ковариационная матрица симметрична: $Var(X)'=Var(X)$.

Конечно же, дисперсию можно записать через математическое ожидание: $Var(X)=E[(X-\E(X))^{2}]=\E(X^{2})-\E(X)^{2}$

Для ковариационной матрицы это выглядит так:
\begin{equation}
Var(\vec{X})=E[(X-\E(X))\cdot (X-\E(X))']=\E(XX')-\E(X)\cdot \E(X)'
\end{equation}

Упр. Убедитесь в этом.


Мы быстренько пройдемся по свойствам этих вещей:

Если $A$ - неслучайная матрица и $b$ - неслучайный вектор подходящих размеров, то:
\begin{enumerate}
\item $\E(A\vec{X}+b)=A\E(\vec{X})+b$
\item $\E(\vec{X}A+b)=\E(\vec{X})A+b$
\item $Var(AX+b)=AVar(X)A'$.
\item $Var(X)$ - неотрицательно определена, имеет $n$ неотрицательных собственных чисел (с учетом кратности), где $n$ размерность вектора $X$.
\end{enumerate}

Проверка этих свойств - еще одно упражнение по линейной алгебре.


\subsubsection*{Краткая схема разных определений}

Есть три с половиной подхода определить многомерное нормальное распределение.

Вектор $\vec{X}$ имеет многомерное нормальное распределение, если выполнено одно из трех равносильных условий:
\begin{enumerate}
\item Вектор $\vec{X}$ представим в виде $\vec{X}=A\vec{Z}+\mu$, где $\vec{Z}$ --- это вектор независимых стандартных нормальных случайных величин
\item Любая линейная комбинация $c_{1}X_{1}+c_{2}X_{2}+...+c_{n}X_{n}$ имеет одномерное нормальное распределение
\item Характеристическая функция вектора $\vec{X}$ имеет вид:
\begin{equation}
\phi(\vec{u})=\exp\left(i\cdot \mu'\cdot u-\frac{1}{2}u'Vu \right)
\end{equation}
\end{enumerate}

В случае $det(V)\neq 0$ первые три формулировки становятся эквивалентны четвертой:

Вектор $\vec{X}$ имеет невырожденное многомерное нормальное распределение, если
\begin{enumerate}
\item Функция плотности вектора $\vec{X}$ имеет вид:
\begin{equation}
p(\vec{x})=(2\pi)^{-n/2}(\det(V))^{-1/2}\cdot \exp\left(-\frac{1}{2}(\vec{x}-\mu) V^{-1} (\vec{x}-\mu)' \right) 
\end{equation}
\end{enumerate}

Мы выбираем первое свойство как определение, а остальные будет доказывать как теоремы.


\subsubsection*{Определение и функция плотности}


Очень часто бывает удобно говорить о константе как о нормально распределенной случайной величине с нулевой дисперсией. Это просто соглашение. Никакой функции плотности у константы конечно же нет! Просто когда мы говорим $N(\mu,\sigma^{2})$, $\sigma>0$, мы имеем ввиду <<честное>> нормальное распределение c функцией плотности
\begin{equation}
p(x)=\frac{1}{\sqrt{2\pi\sigma^{2}}}\exp\left(-\frac{(x-\mu)^{2}}{2\sigma^{2}}\right)
\end{equation}
, а когда говорим $N(\mu,0)$ имеем ввиду просто константу $\mu$.

Т.е. говоря <<нормальное распределение>> мы теперь можем иметь ввиду и константу. Если же нам надо подчеркнуть, что речь идет о <<честном>> нормальном распределении с функцией плотности, то мы будем говорить <<невырожденное нормальное распределение>>.


Теперь мы можем дать несколько необычное определение одномерной нормальной величины... Пусть $a$ и $b$ - это константы.

\begin{mydef}
Если случайная величина $X$ представима в виде $X=aZ+b$, где $Z\sim N(0,1)$, то мы говорим, что $X$ имеет \indef{нормальное распределение} $N(b,a^{2})$. Если, кроме того, $a\neq 0$, то мы говорим, что $X$ имеет \indef{невырожденное нормальное распределение}.
\end{mydef}


Теперь мы готовы сказать, что такое многомерное нормальное распределение. Пусть матрица $A$ - неслучайная, вектор $b$ - неслучайный. Размерности подходящие.

\begin{mydef} \label{def:mult_norm}
Если $X=AZ+b$, где вектор $Z$ состоит из независимых $Z_{i}\sim N(0;1)$, то мы говорим, что вектор $X$ имеет \indef{нормальное распределение} $N(b,AA')$. Если, кроме того, $A$ обратимая матрица, $det(A)\neq 0$, то мы говорим, что $X$ имеет \indef{невырожденное нормальное распределение}.
\end{mydef}

В этой главе и в книге в целом мы \indef{постараемся} уточнять, но обычно из контекста понятно, включает ли автор в понятие нормального распределения вырожденные случаи. Как правило, да, включает.


\begin{myex}
Вектор $X$ имеет вырожденное нормальное распределение, у него нет двумерной функции плотности:

\begin{equation}
\left(\begin{array}{c} X_{1} \\ X_{2}  \end{array} \right)=
\left(\begin{array}{cc} 0 & 1 \\ 0 & 1  \end{array} \right)\cdot
\left(\begin{array}{c} Z_{1} \\ Z_{2}  \end{array} \right)=
\left(\begin{array}{c} Z_{1} \\ Z_{1}  \end{array} \right)
\end{equation}

\end{myex}



Невырожденный нормальный вектор имеет функцию плотности...
\begin{myth}
Вектор $X$ имеет невырожденное нормальное распределение $N(b,AA')$, если и только если его функция плотности имеет вид 
\begin{equation}
p(\vec{x})=(2\pi)^{-n/2}(\det(AA'))^{-1/2}\cdot \exp\left(-\frac{1}{2}(\vec{x}-b) (AA')^{-1} (\vec{x}-b)' \right) 
\end{equation}
Если не обращать внимания на константу, которая нужна чтобы интеграл под функцией плотности равнялся единице, то:
\begin{equation}
p(\vec{x})\sim \exp\left(-\frac{1}{2}(\vec{x}-b) (AA')^{-1} (\vec{x}-b)' \right) 
\end{equation}
\end{myth}

\begin{proof}


\end{proof}



\subsubsection*{Напоминалка про характеристические функции}


Опытный охотник увидев уши, торчащие из кустов, скажет: <<Ага, это заяц!>> Для этих же целей нужна и характеристическая функция. Она позволяет опознать закон распределения случайной величины. 

Узнав например, что характеристическая функция случайной величины $W$ равна $\phi(u)=\frac{\exp(iu)-1}{iu}$, опытный охотник скажет: <<Ага, это равномерная на $[0;1]$!>>. Увидев характеристическую функцию $\phi(u)=(p\cos(u)+ip\sin(u)+1-p)^{n}$, опытный охотник узнает биномиальное распределение $Bin(n,p)$.

Настала пора для формального определения:

\begin{mydef}
Характеристической функцией случайной величины $W$ называется функция $\phi:\mathbb{R}\to \mathbb{C}$:
\begin{equation}
\phi_{W}(u):=\E(\cos(uW))+i\E(\sin(uW))
\end{equation}
Или более кратко:
\begin{equation}
\phi_{W}(u):=\E(\exp(iuW))
\end{equation}
\end{mydef}

Косинус и синус - ограниченные функции, поэтому для любой случаной величины $W$ существуют $\E(\sin(uW))$ и $\E(\cos(uW))$. А значит и характеристическая функция всегда существует.
Тот факт, что по характеристической функции можно узнать закон распределения строго формулируется в виде теоремы:


\begin{myth}
Пусть $X_{1}$ и $X_{2}$ --- две случайной величины. Функции распределения $F_{1}(t)$ и $F_{2}(t)$ совпадают если и только если совпадают характеристические функции $\phi_{1}(u)$ и $\phi_{2}(u)$
\end{myth}









Для описания многомерных нормальных величин характеристическая функция оказывается более удобна, чем функция плотности. Дело в том, что у вырожденных нормальных векторов функции плотности нет, а характеристическая функция существует всегда.



\subsubsection*{Еще два эквивалентных определения}


\begin{myth}
Вектор $X$ имеет нормальное распределение $N(b,AA')$, возможно вырожденное, если и только если его характеристическая функция имеет вид 
\begin{equation}
\phi(\vec{u})=\exp\left(i\cdot \mu'\cdot u-\frac{1}{2}u'Vu \right)
\end{equation}

\end{myth}



Важная особенность многомерного нормального распределения.

\begin{myth} \label{th:normal_one2mult}
Вектор $X=(X_{1},...,X_{n})$ имеет многомерное нормальное распределение (возможно вырожденное) если и только если любая линейная комбинация $Y=c_{1}X_{1}+c_{2}X_{2}+...+c_{n}X_{n}$ имеет одномерное нормальное распределение (возможно вырожденное).
\end{myth}

\begin{proof}
Если $X$ - многомерное нормальное, то $X=AZ+b$. Следовательно, $Y=c_{1}X_{1}+c_{2}X_{2}+...+c_{n}X_{n}=\vec{c}X=\vec{c}AZ+\vec{c}b$. Значит $Y$ по определению \ref{def:mult_norm} имеет нормальное распределение.

Если любая линейная комбинация нормальна...

\end{proof}






\subsubsection*{Стандартизация}

В одномерном случае, если $Var(X)\neq 0$, то случайную величину $X$ можно <<стандартизировать>>, т.е. превратить в случайную величину с нулевым средним и единичной дисперсией: 

\begin{equation}
Z:=\frac{X-\E(X)}{\sqrt{Var(X)}}
\end{equation}

У случайной величины $Z$: $\E(Z)=0$, $Var(Z)=1$.

В многомерном случае, если матрица $Var(X)$ обратима, то случайный вектор $\vec{X}$ можно <<стандартизовать>>, т.е. превратить в вектор некоррелированных случайных величин с нулевым средним и единичной дисперсией.

\begin{equation}
\vec{Z}:=(Var(X))^{-1/2}(\vec{X}-\E(\vec{X}))
\end{equation}

Для этой операции нужно понимать, что такое $A^{\frac{1}{2}}$. Для положительно определенной матрицы эта операция корректна. Если $A$ положительно определена, то у нее есть разложение $A=PDP^{-1}$ причем $D$ - диагональная матрица, где на главной диагонали стоят положительные собственные значения $A$. И, стало быть, $A^{\frac{1}{2}}=PD^{\frac{1}{2}}P^{-1}$.

Если нужно решить какую-то задачу, связанную с многомерным нормальным распределением, то стандартизация может здорово облегчить вычисления. Даже в вырожденном случае имеет смысл перейти к рассмотрению независимых нормальных $N(0;1)$ случайных величин.
\begin{myex}
Пусть рост любой женщины имеет распределение $N(165,25)$. А корреляция между ростом матери и ростом дочери равна $\rho$. Какова вероятность того, что дочь высокая (выше среднего роста), если мама - высокая?

Начинаем решать:

Обозначим $X_{1}$ - рост матери, $X_{2}$ - рост дочери. $\P(X_{2}>165|X_{1}>165)=\frac{\P(X_{1}>165\cap X_{2}>165)}{\P(X_{1}>165)}=2\P(X_{1}>165\cap X_{2}>165)$.

Если решать без стандартизации, <<в лоб>>. Вылезает интеграл, который берется усердным трудом...

Если перейти к стандартным $Z_{1}$ и $Z_{2}$...


%\left(\begin{array}{c} X_{1} \\ X_{2} \end{array} \right)\sim 

\end{myex}


\subsubsection*{Многомерное --- это больше чем несколько одномерных}

Из теоремы \ref{th:normal_one2mult} следует в частности, что:

\begin{myth}
Если вектор $\vec{X}$ имеет многомерное нормальное распределение, то и любая его компонента $X_{i}$ имеет нормальное распределение. Кроме того, если распределение $\vec{X}$ --- невырожденное, то и распределение каждой компоненты $X_{i}$ невырожденное.
\end{myth}

Обратное утверждение неверно. Мы приведем пример, в котором $X_{1}$ и $X_{2}$ имеют нормальное распределение по отдельности, но не имеют совместного нормального распределения.

\begin{myex}
Пусть $X_{1}\sim N(0;1)$, а $K$ --- это случайная величина равновероятно принимающая значения $1$ и $-1$, причем $K$ и $X_{1}$ независимы. Определим
\begin{equation}
X_{2}=K\cdot |X_{1}|
\end{equation}
Проверим, что $X_{2}$ имеет нормальное распределение. Для положительных $x$:
\begin{multline}
\P(X_{2}\leq x)=\P(K=-1)+\P(K=1)\cdot \P(|X_{1}|\leq x)=\\
=0.5+0.5\P(|X_{1}|\leq x)=0.5+\P(X_{1}\in [0;x])=\P(X_{1}\leq x)=F(x)
\end{multline}
Для отрицательных $x$ проверьте сами!

Можно заметить, что $X_{1}+X_{2}$ не является нормально распределенной случайной величиной. Более того, распределение $X_{1}+X_{2}$ не является непрерывным. Действительно,
\begin{equation}
\P(X_{1}+X_{2}=0)=\P(X_{1}+K|X_{1}|=0)=\P(K\neq \sgn(X_{1}))=0.5
\end{equation}
Если бы вектор $(X_{1}, X_{2})$ имел совместное нормальное распределение, то сумма $X_{1}+X_{2}$ была бы нормально распределенной.
Кстати, найдем ковариацию между $X_{1}$ и $X_{2}$:
\begin{multline}
Cov(X_{1},X_{2})=\E(X_{1}X_{2})-\E(X_{1})\E(X_{2})=\E(X_{1}X_{2})=\E(X_{1}\cdot K\cdot|X_{1}|)=\\
=\E(K)\E(X_{1}\cdot |X_{1}|)=0
\end{multline}
Значит наши $X_{1}$ и $X_{2}$ были некоррелированны, нормальны по отдельности. Вместе с тем они не были нормальны в совокупности. И конечно, они зависимы, т.к. $|X_{1}|=|X_{2}|$.

\end{myex}


\subsubsection*{Некоррелированность и независимость}
% включена в абзац про связь одномерного и многомерного

Для многомерного нормального распределения некоррелированность равносильна независимости.

Доказательство для двумерного...





\subsubsection*{Условное распределение}

Предположим, что вектор $(X,Y)$ имеет совместное нормальное распределение. Часто возникает задача прогнозирования случайной величины $Y$, если значение случайной величины $X$ известно.

%Для начала мы отметим, что при известном $X$ условное распределение $Y$ также будет нормальным.

Для удобства вычислений всегда используем стандартизацию! Вместо исходных $X$ и $Y$ рассмотрим:
\begin{equation}
Z_{1}=\frac{X-\E(X)}{\sigma_{X}}, \quad Z_{2}=\frac{Y-\E(Y)}{\sigma_{Y}}
\end{equation}
Естественно, $Z_{1}\sim N(0;1)$ и $Z_{2}\sim N(0;1)$ и $Corr(Z_{1},Z_{2})=Corr(X,Y)$.

Найдем условную функцию плотности $Z_{2}$ при известном $Z_{1}$:
\begin{equation}
p(z_{2}|z_{1})=\frac{p(z_{1},z_{2})}{p(z_{1})}
\end{equation}

Поскольку нас интересует только зависимость от $z_{2}$, то на $p(z_{1})$ можно не обращать внимания. Значком $\sim$ мы будем обозначать равенство с точностью до константы не зависящей от $z_{2}$:
\begin{multline}
p(z_{2}|z_{1})\sim p(z_{1},z_{2})\sim \exp\left(-\frac{1}{2}(z_{1},z_{2})\left(\begin{matrix}
1 & \rho \\ 
\rho & 1
\end{matrix}\right)^{-1}\left( \begin{matrix}
z_{1} \\ 
z_{2}
\end{matrix}\right)   \right)=\\
=\exp\left(-\frac{1}{2(1-\rho^{2})}(z_{1},z_{2})\left(\begin{matrix}
1 & -\rho \\ 
-\rho & 1
\end{matrix}\right)\left( \begin{matrix}
z_{1} \\ 
z_{2}
\end{matrix}\right)   \right)= \\
=\exp\left(-\frac{1}{2(1-\rho^{2})}\left(z_{2}^{2}-2\rho z_{1}z_{2}+z_{1}^{2}\right) \right)\sim 
\exp\left(-\frac{1}{2(1-\rho^{2})}\left(z_{2}-\rho z_{1}\right)^{2}\right)
\end{multline}

Сравним полученный результат с функцией плотности одномерного нормального распределения:
\begin{equation}
p(x)\sim\exp\left(-\frac{(x-\mu)^{2}}{2\sigma^{2}}\right)
\end{equation}

Замечаем, что $\sigma^{2}=1-\rho^{2}$ и $\mu=\rho z_{2}$. И оформляем результат вычислений в виде теоремы:
\begin{myth}
Если $Z_{1}$ и $Z_{2}$ имеют совместное нормальное распределение, $\E(Z_{i})=0$, $Var(Z_{i})=1$ и $Corr(Z_{1},Z_{2})=\rho$, то условное распределение $Z_{1}$ при известном $Z_{2}$ является нормальным и:
\begin{equation}
\E(Z_{2}|Z_{1}=z)=\rho\cdot z
\end{equation}
\begin{equation}
Var(Z_{2}|Z_{1}=z)=1-\rho^{2}
\end{equation}
\end{myth}

Формулы дают нам смысл:
\begin{enumerate}
\item Корреляция стандартизированных нормальных величин показывает на сколько в среднем растет одна случайная величина при росте другой на единицу.
\item Условная дисперсия $Z_{2}$ не зависит от конкретного значения $Z_{1}$. Это несколько неожиданно, но интуитивного объяснения я не знаю.
\end{enumerate}
 

Для возврата к исходным $X$ и $Y$ замечаем, что условие $X=x$ равносильно тому, что $Z_{1}=\frac{x-\mu_{x}}{\sigma_{x}}$ и, кроме того, $Y=\mu_{y}+\sigma_{y}Z_{2}$.

Сделав обратную замену, получаем:
\begin{myth}
Если $X$ и $Y$ имеют совместное нормальное распределение, $X\sim N(\mu_{x},\sigma^{2}_{x})$, $Y\sim N(\mu_{y},\sigma^{2}_{y})$ и $Corr(X,Y)=\rho$, то условное распределение $Y$ при известном $X$ является нормальным и:
\begin{equation}
\E(Y|X=x)=\mu_{y}+\rho\sigma_{y}\frac{x-\mu_{x}}{\sigma_{x}}
\end{equation}
\begin{equation}
Var(Y|X=x)=(1-\rho^{2})\sigma_{y}^{2}
\end{equation}
\end{myth}

Получаем трактовку корреляции: Если $X$ и $Y$ имеют совместное нормальное распределение, то корреляция  показывает на сколько своих стандартных отклонений в среднем растет $Y$  при росте $X$ на одно свое стандартное отклонение. 


\subsubsection*{Геометрический смысл ковариационной матрицы}


\begin{mydef}
Матрица $R$ называется \indef{матрицей поворота} если одновременно выполнены два условия:
$R'R=I$ и $\det(R)=1$
\end{mydef}

Почему определение поворота именно такое?

Условие $RR'=I$ означает две вещи:
\begin{enumerate}
\item Вектор $R\vec{x}$ имеет такую же длину, как и вектор $\vec{x}$:

\begin{equation}
|\vec{x}|^{2}=\vec{x}'\vec{x}=\vec{x}'R'R\vec{x}=(R\vec{x})'(R\vec{x})=|R\vec{x}|^{2}
\end{equation}

\item Угол между $\vec{x}$ и $\vec{y}$ равен углу между $R\vec{x}$ и $R\vec{y}$
\begin{equation}
\cos(\vec{x},\vec{y})=\frac{\vec{x}'\vec{y}}{|\vec{x}||\vec{y}|}=
\frac{\vec{x}'R'R\vec{y}}{|R\vec{x}||R\vec{y}|}=
\frac{(R\vec{x})'(R\vec{y})}{|R\vec{x}||R\vec{y}|}=\cos(R\vec{x},R\vec{y})
\end{equation}
\end{enumerate}

Условию $R'R=I$ подходят матрицы с определителем $\det(A)=\pm 1$. Дополнительное  условие $\det(R)=1$ исключает <<отражения>>.


Упражнение. Мы доказали, что матрицы вида $R'R=I$ сохраняют углы и длины. Докажите, что никакие другие матрицы не сохраняют одновременно углы и длины.

Решение. Рассмотрим вектор $e_{k}=(0,0,\ldots,0,1,0,\ldots,0)'$. 

Из линейной алгебры:
\begin{myth}
Если $A_{n\times n}$ действительная симметричная положительно полу-определенная матрица, то
\begin{enumerate}
\item У $A$ имеется ровно $n$ действительных собственных чисел
\item $A$ представима в виде 
\begin{equation}
A=RDR^{-1}=RDR'
\end{equation}, 
где $D$ --- диагональная матрица из собственных чисел матрицы $A$, а $R$ --- матрица поворота из собственных векторов матрицы $A$.
\end{enumerate}
\end{myth}


Для начала сформулируем геометрические факты:
\begin{mydef}
Множество точек называется \indef{эллипсоидом}, если оно задается системой уравнений
\begin{equation}
(\vec{x}-\vec{x}_{0})' A (\vec{x}-\vec{x}_{0})=1
\end{equation}
, где $A$ --- положительно определенная матрица. Точка $\vec{x}_{0}$ --- центр эллипсоида. В двумерном случае эллипсоид называют \indef{эллипсом}.
\end{mydef}


Почему определение эллипса именно такое?

Матрицу $A$ можно представить в виде $A=R'DR$, где $R$ --- матрица поворота, а $D$ --- диагональная матрица собственных чисел матрицы $A$. Отсюда получаем, что уравнение эллипса можно записать в виде:
\begin{equation}
(R(\vec{x}-\vec{x}_{0}))' D (R(\vec{x}-\vec{x}_{0}))=1
\end{equation}
Если ввести обозначения $\vec{z}=R(\vec{x}-\vec{x}_{0})$, то уравнение примет вид:
\begin{equation}
\sum_{i=1}^{n} d_{ii}z_{i}^{2}=1
\end{equation}

Именно в силу этого представления:
\begin{mydef}
Для эллипса $(\vec{x}-\vec{x}_{0})' A (\vec{x}-\vec{x}_{0})=1$ собственные векторы матрицы $A$ называют \indef{направлениями полуосей}. Если $\lambda_{i}$ --- это собственное число матрицы $A$, то величины $1/\sqrt{\lambda_{i}}$ называют \indef{длинами полуосей}.
\end{mydef}


Теперь мы готовы нарисовать линии уровня многомерного нормального распределения!
\begin{myth}
Для невырожденного нормального распределения линии уровня функции плотности являются эллипсоидами. Направления главных осей задаются собственными векторами ковариационной матрицы. Соотношение длин полуосей равно соотношению корней из собственных чисел ковариационной матрицы.
\end{myth}


\begin{proof}

Для доказательства нам потребуется технический факт из линейной алгебры:
\begin{myth}
Eсли $\vec{a}$ собственный вектор для матрицы $V$ с собственным числом $\lambda$, то $\vec{a}$ собственный вектор для матрицы $V^{-1}$ с собственным числом $1/\lambda$.
\end{myth}
\begin{proof}
Если $\vec{a}$ собственный вектор матрицы $V$, то с одной стороны:
\begin{equation}
V^{-1}\cdot (V\cdot \vec{a})=V^{-1} (\lambda \vec{a})=\lambda V^{-1}\vec{a}
\end{equation}

С другой стороны:
\begin{equation}
(V^{-1}\cdot V)\cdot \vec{a}=\vec{a}
\end{equation}

Т.е. $\lambda V^{-1}\vec{a}=\vec{a}$. Или:
\begin{equation}
V^{-1}\vec{a}=\frac{1}{\lambda}\vec{a}
\end{equation}

\end{proof}


Итак, пусть $X\sim N(\vec{\mu};V)$.

Тогда условие
\begin{equation}
p(\vec{x})=const
\end{equation}
после преобразований равносильно тому, что:
\begin{equation}
(\vec{x}-\vec{\mu})\cdot V^{-1}\cdot (\vec{x}-\vec{\mu})'=const
\end{equation}

Т.е. мы получили наше определение эллипсоида с $A=\frac{1}{const}V^{-1}$.

Направления полуосей задаются собственными векторами $A$, значит они совпадают с собственными векторами $V$. 

Длины полуосей обратно пропорциональны корням из собственных чисел $A$, значит они прямо пропорциональны корням из собственных чисел $V$.
\end{proof}


\begin{myex}
Пусть $X\sim N(0;V)$ и 
\begin{equation}
V=
\left(
\begin{array}{cc}
5 & 1 \\ 
1 & 9
\end{array} 
\right)
\end{equation}

Нарисуйте линии уровня функции плотности $p(x_{1},x_{2})$


Решение.

Тут обязательно картинки. Эллипс. Главные оси. С кодом R и Sage!

\end{myex}


В случае независимых нормальных случайных величин с одинаковой дисперсией линиями уровня будут окружности (сферы при более высоких размерностях). Поскольку поворот никак не влияет на линию уровня мы бесплатно получаем следующую теорему:

\begin{myth} \label{th:rotate_normal}
Если матрица $R$ --- это матрица поворота и вектор $\vec{Z}\sim N(\vec{0},I)$ то вектор $R\vec{Z}\sim N(\vec{0},I)$
\end{myth}



\subsubsection*{Хи-квадрат распределение}

Как известно,

\begin{mydef} Если случайная величина $W$ представима в виде $W=\sum_{i=1}^{k}Z_{i}^{2}$, где $Z_{i}$ --- независимые стандартные нормальные случайные величины, то говорят, что $W$ имеет \indef{хи-квадрат распределение c $k$ степенями свободы}.
\end{mydef}


К сожалению, проверять, что что-то имеет хи-квадрат распределение напрямую очень неудобно.

\begin{myex}
Давайте попробуем по определению показать, что если $X_{i}\sim N(\mu;\sigma^{2})$ и независимы, то 
\begin{equation}
\frac{1}{\sigma^{2}}\sum_{i=1}^{n}(X_{i}-\bar{X}_{n})^{2}\sim \chi_{n-1}^{2}
\end{equation}

Как всегда, сначала стандартизируем наши $X_{i}$. 



...


\end{myex}


Гораздо более удобным оказывается следующий способ:
\begin{myth}
Если вектор $\vec{Z}$ состоит из независимых стандартных нормальных случайных величин и все собственные числа симметричной матрицы $A$ равны либо нулю, либо единице, то 
\begin{equation}
\vec{Z}'A\vec{Z}\sim \chi_{r}^{2}
\end{equation}
где $r$ --- количество собственных чисел матрицы $A$ равных единице.
\end{myth}

\begin{proof}
Матрица $A$ представима в виде $A=R'DR$. Поэтому:
\begin{equation}
\vec{Z}'A\vec{Z}=\vec{Z}'R'DR\vec{Z}=(R\vec{Z})'D(R\vec{Z})
\end{equation}

Остается заметить, что $R\vec{Z}$ --- это вектор независимых стандартных нормальных случайных величин в силу теоремы \ref{th:rotate_normal}. Если обозначить этот вектор буквой $\vec{W}=R\vec{Z}$, то
\begin{equation}
\vec{Z}'A\vec{Z}=\vec{W}'D\vec{W}=\sum_{i=1}^{n}d_{ii}W_{i}^{2}
\end{equation}

Матрица $D$ --- это матрица собственных чисел матрицы $A$, т.е. $d_{ii}$ равны либо нулю, либо единице. Получается, что $\vec{Z}'A\vec{Z}$ --- это сумма $r$ стандартных независимых нормальных случайных величин.
\end{proof}


\begin{myth}
Собственные числа симметричной матрицы $A$ равны либо нулю, либо единице, если и только если  $A^{2}=A$
\end{myth}

\begin{proof}
С одной стороны:
\begin{equation}
A^{2}=(R'DR)(R'DR)=R'D(RR')DR=R'DDR=R'D^{2}R
\end{equation}
С другой стороны $A=R'DR$. Значит $A^{2}=A$ если и только если $D^{2}=D$. Но $D$ --- диагональная матрица, поэтому условие $D^{2}=D$ равносильно тому, что на диагонали стоят либо нолики, либо единички.
\end{proof}


\begin{myex}
Тот же пример только быстрее...
...

\end{myex}


\subsubsection*{Границы на хвостовые вероятности}
При изучении броуновского движения полезны два неравенства.

\begin{myth} Если $Z\sim N(0;1)$, то:
\begin{equation}
\frac{1}{\sqrt{2\pi}}\frac{\exp(-x^{2}/2)}{x+x^{-1}}\leq \P(Z\geq x)\leq \frac{1}{\sqrt{2\pi}}\frac{\exp(-x^{2}/2)}{x}, \quad x>0
\end{equation}
\end{myth}

\begin{proof} Заметим, что $\exp(-t^{2}/2)'=-t\exp(-t^{2}/2)$. Получаем верхнюю границу:
\begin{multline}
\P(Z\geq x)=\int_{x}^{\infty} \frac{1}{\sqrt{2\pi}}\exp(-t^{2}/2)dt\leq \\
\leq \int_{x}^{\infty} \left(\frac{t}{x} \right)\frac{1}{\sqrt{2\pi}}\exp(-t^{2}/2)dt=
\frac{1}{\sqrt{2\pi}}\frac{\exp(-x^{2}/2)}{x}
\end{multline}
Заметим, что $(t^{-1}\exp(-t^{2}/2))'=(1+t^{-2})\exp(-t^{2}/2)$. Получаем нижнюю границу:
\begin{multline}
\P(Z\geq x)=\int_{x}^{\infty} \frac{1}{\sqrt{2\pi}}\exp(-t^{2}/2)dt
\geq \int_{x}^{\infty} \left(\frac{1+t^{-2}}{1+x^{-2}} \right)\frac{1}{\sqrt{2\pi}}\exp(-t^{2}/2)dt= \\
=\frac{1}{\sqrt{2\pi}}\frac{1}{1+x^{-2}}\frac{\exp(-x^{2}/2)}{x}=
\frac{1}{\sqrt{2\pi}}\frac{\exp(-x^{2}/2)}{x+x^{-1}}
\end{multline}

\end{proof}

\begin{myth} Если $Z\sim N(0;1)$, то:
\begin{equation}
\frac{2x}{\sqrt{2\pi e}}\leq \P(|Z|\leq x)\leq \frac{2x}{\sqrt{2\pi}}, \quad 0<x\leq 1
\end{equation}
\end{myth}

\begin{proof}
Если $|t|\leq 1$, то
\begin{equation}
\frac{1}{\sqrt{2\pi e}}\leq \frac{1}{\sqrt{2\pi}}\exp(-t^{2}/2)\leq \frac{1}{\sqrt{2\pi}}
\end{equation}
Интегрируя это неравенство от $-x$ до $x$ получаем требуемое.
\end{proof}


Еще кусок из блога... (ссылка?) ...

\subsubsection*{Откуда взялось $\pi$ в формуле?}

Есть много способов объяснить, откуда берется $\pi$ в формуле функции плотности. Вот то, которое нравится мне.

Рассмотрим пару независимых стандартных нормальных случайных величин $X$ и $Y$. Их функция плотности имеет вид:
\begin{equation}
p(x,y)=c\cdot \exp\left( -\frac{1}{2} (x^{2}+y^{2})\right)
\end{equation}

Рассмотрим более подробно функцию
\begin{equation}
f(x,y)=\exp\left( -\frac{1}{2} (x^{2}+y^{2})\right)
\end{equation}

Значение $f(x,y)$ зависит только от расстояния от точки $(x,y)$ до начала координат. Значит объем под <<шляпой>> является фигурой вращения.

Картинка (слева и справа повернутая):

...

Уже понятно, что $\pi$ --- в деле. Остается вспомнить, что объем фигуры вращения определяется по формуле $Vol=\int_{a}^{b}\pi r^{2}(t)dt$.

В нашем случае: $a=0$, $b=f(0,0)=1$, а $r(t)$ находится из условия:
\begin{equation}
\exp\left( -\frac{1}{2} r^{2}(t)\right)=t
\end{equation}

Находим $r^{2}(t)$ и получаем:
\begin{equation}
r^{2}(t)=-2\ln(t)
\end{equation}

Находим объем фигуры вращения:
\begin{equation}
Vol=\int_{0}^{1} \pi r^{2}(t)dt=\int_{0}^{1} \pi (-2\ln(t))dt=-2\pi\int_{0}^{1}\ln(t)dt=2\pi 
\end{equation}

Интеграл от $\ln(t)$ можно взять либо по частям, либо заметив, что:
\begin{equation}
\int_{0}^{1}\ln(t)dt=-\int_{0}^{\infty}\exp(-t)dt
\end{equation}
Картинка:


Настоящие знатоки берут интеграл $\int_{0}^{\infty}\exp(-t)dt$ по методу Мамикона Мнацаканяна \cite{apostol:visual_calculus} в уме глядя на картинку:



Пояснение к картинке:

Производная функции $\exp(-x)$ равна ей самой умноженной на минус один. Поэтому тень от касательной всегда имеет длину один. Интересующая нас площадь равна площади треугольника плюс оставшаяся площадь. Касательные заметающие оставшуюся площадь можно перенести так, чтобы они замели треугольник. Значит интеграл равен удвоенной площади треугольника.

 
Мы доказали, что объем под функцией $f(x,y)$ равен $2\pi$. Следовательно, функция плотности $p(x,y)$ должна иметь вид:
\begin{equation}
p(x,y)=\frac{1}{2\pi}\cdot \exp\left( -\frac{1}{2} (x^{2}+y^{2})\right)
\end{equation}

Для независимых величин $p(x,y)=p(x)p(y)$, следовательно:
\begin{equation}
p(x)=\frac{1}{\sqrt{2\pi}}\cdot \exp\left( -\frac{1}{2} x^{2}\right)
\end{equation}



\subsubsection*{ЦПТ --- доказательство через характеристические функции}

\begin{myth}
Пусть $X_{n}$ --- последовательность случайных величин с характеристическими функциями $\phi_{n}(u)$. Пусть кроме того, $X$ --- случайная величина с характеристической функцией $\phi(u)$. Последовательность $X_{n}$ сходится по распределению к случайное величине $X$ если и только если последовательность функций $\phi_{n}(u)$ сходится поточечно к функции $\phi(u)$.
\end{myth}




%\part{Теорема Берри-Эссена}
%\input{../berry_essen.tex}


%\part{Две геометрии случайных величин}
%\input{geom_random.tex}


%\part{Теорема Дуба}


% про теорему о моменте остановки
% сначала в дискретном времени, потом в непрерывном

% идеология: лучше сделать лишний повтор, ибо: 1 --- можно рассказать несколько раз, не опасаясь, что кто-нибудь поймет :), 2 --- появляется большая независимость глав.

% аудитория: студенты-нематематики, хотя и студенты-математики могут найти что-нибудь новое для себя


\section{Теорема Дуба}

%На стиль изложения во многом повлияли источники \cite{stirzaker:prp}, \cite{chang:sp}, \cite{ross:scp}

Пусть $X_{t}$ --- наше благосостояние в справедливой игре в момент времени $t$ или мартингал. Наша стратегия заключается в том, чтобы в нужный момент завершить игру, скажем, после крупного выигрыша. Другими словами, наша стратегия определяется моментом остановки $T$. Этот момент остановки --- случайная величина, так как может зависить от хода игры. Случайной величиной является также и $X_{T}$ --- выигрыш на момент выхода из игры. Вопрос в том, чему равен средний выигрыш на момент прекращения игры, $\E(X_{T})$?

Ответ дает теорема Дуба:
Если не ждать <<слишком>> долго, то каким бы хитрым не был момент остановки ожидаемый выигрыш будет равен начальной сумме, $\E(X_{T})=\E(X_{1})$. 

Сразу приведем пример <<слишком>> долгого ожидания: ждать до выигрыша в один рубль в классическом случайном блуждании. Условие  $\E(X_{T})=\E(X_{1})$ здесь нарушено: выигрыш на момент выхода из игры равен одному рублю (по построению), а стартовая сумма равна нулю. Почему это слишком долгое ожидание? Потому, что в этом случае можно доказать, что $\E(T)=+\infty$. 

Точный смысл понятия <<слишком>> долго можно увидеть в теореме: 

Если $X_{t}$ --- мартингал, $T$ --- момент остановки, и выполнено хотя бы одно из пяти условий: 

(i) Момент $T$ ограничен, то есть существует число $M$, такое что $T<M$. 

(ii) $\P(T<+\infty)=1$ и процесс $X_{t\wedge T}$ ограничен, то есть существует число $M$, такое что для любого $t$ верно неравенство $|X_{t\wedge T}|<M$. 

(iii) $\E(T)<+\infty$ и существует число $M$, такое что для любого $t$ верно неравенство $\E(|X_{t+1}-X_{t}||\mathcal{F}_{n})<M$.

(iv) $\P(T<+\infty)=1$, $\E(|X_{T}|)<\infty$ и $lim_{t\to\infty}\E(X_{t}1_{T>t})=0$.

(v) $\P(T<+\infty)=1$ и мартингал $X_{t}$ является равномерно интегрируемым.

То: $\E(X_{T})=\E(X_{1})$.

В большинстве случаев первых трех критериев достаточно для практического применения.
Четвертый критерий является следствием любого из первых трех (?).

В пятом критерии используется определение равномерной интегрируемости...
Набор случайных величин является равномерно интегрируемым, если...


Из этих критериев все кроме третьего (может есть какой-то <<предельный>> аналог и третьего? --- я не знаю) работают в непрерывном времени.

Из теоремы Дуба легко вывести тождество Вольда. Wald's identity.
 
Если складывать случайное количество случайных величин (опять же, не <<слишком>> много), то среднее значение суммы равно произведению среднего размера слагаемых на среднее количество слагаемых. Это кажется очевидным, но на самом деле не все так просто, так теорема допускает, что количество слагаемых может зависить от размера слагаемых.

Если $X_{i}$ --- независимые одинаково распределенные случайные величины и $T$ --- случайная величина принимающая целые неотрицательные значения и $\E(T)<\infty$, то $\E(\sum_{i=1}^{T}X_{i})=\E(X_{i})\E(T)$.

Доказательство:

Случаный процесс $M_{t}=\sum_{i=1}^{t}X_{i}-t\cdot \E(X_{i})$ --- мартингал. Применяя к нему теорему Дуба получаем, что $\E(M_{T})=\E(M_{1})$. Здесь $\E(M_{1})=0$, $M_{T}=\sum_{i=1}^{T}X_{i}-T\cdot \E(X_{i}$, поэтому $\E(\sum_{i=1}^{T}X_{i})=\E(X_{i})\E(T)$.




\section{Примеры}

Между примерами и задачами нет существенного различия. Если легко, то можно взять условие примера и попробовать его решить как задачу. А если тяжело, то можно посмотреть решение задачи. Пожалуй, тексты примеров в каком-то смысле <<классические>>. Иногда используемые для решения мартингалы настолько изящны, что вспоминаются субтитры при показе сложных трюков по телевизору: <<Трюки выполнены профессиональными каскадерами. Не пробуйте повторить их самостоятельно>>. Только здесь нужно пробовать!

% включить доказательство без теоремы Дуба


Задача. Симметричное случайное блуждание.

Улитка начинает свой путь в точке 0 и за каждую минуту равновероятно смещается влево или вправо на один сантиметр\footnote{Скорость виноградной улитки около 4 см в минуту, но у нас улитка попалась ленивая.}. Справа от улитки на расстоянии $a$ находится виноградное дерево, слева на расстоянии $b$ --- шелковица.

Какова вероятность того, что улитка доползет до виноградного дерева раньше? Сколько в среднем времени ей потребуется чтобы доползти до любого из деревьев?

Решение с мартингалами: $X_{t}$ --- координата улитки в момент времени $T$ --- мартингал. Пусть $T$ --- момент достижения одной из границ ($a$ или $-b$), тогда $|X_{t\wedge T}|\leq \max\{a,b\}$. Момент времени $T$ можно смажорировать геометрической случайной величиной. Разобьем время на интервалы по $(a+b)$ шагов. Будем засчитывать выход за пределы множества $(-b;a)$ только на границах этих интервалов. Очевидно, что требуемое для выхода время от этого только возрастет. 

За каждый интервал вероятность выхода случайного блуждания за пределы множества $(-b;a)$ выше чем $(1/2)^{a+b}$ (это вероятность того, что случайное блуждание сделает $(a+b)$ шагов вправо). Значит требуемое для выхода количество временных интервалов можно ограничить сверху случайной величиной $M$ с геометрическим распределением и вероятностью успеха равной $p=(1/2)^{a+b}$. Получаем, что $\P(T=\infty)\leq \P(M=\infty)=0$ и $\E(T)\leq (a+b) \E(M)=(a+b) 2^{a+b}$.

Применяя пункт (ii) теоремы Дуба получаем, что $\E(X_{T})=\E(X_{1})$. По условию $\E(X_{1})=0$. Величина $X_{T}$ может принимать только два значения $a$ или $-b$. Пусть значение $a$ принимается с вероятностью $p$. Тогда $pa+(1-p)(-b)=0$ и $p=\frac{b}{a+b}$.

Заметим также, что $M_{t}=X^{2}_{t}-t$ --- также мартингал: $\E(M_{t+1}-M_{t}|\mathcal{F}_{t})=\E(2X_{t}\Delta X_{t+1}+(\Delta X_{t+1})^{2}-1|\mathcal{F}_{t})=2X_{t}\E(\Delta X_{t+1}|\mathcal{F}_{t})+\E((\Delta X_{t+1})^{2})-1=0+1-1=0$. 

Применяя ??? пункт (iii) теоремы Дуба получаем, что $\E(M_{T})=\E(M_{1})$. При этом $\E(M_{1})=0$, а $M_{T}=X_{T}^{2}-T$, значит $\E(X_{T}^{2})=\E(T)$. Величина $X_{T}^{2}$ принимает только два значения: $a^{2}$ и $b^{2}$, значит $\E(T)=a^{2}\frac{b}{a+b}+b^{a}\frac{a}{a+b}=ab$.

Задача. Несимметричное случайное блуждание 

Улитка начинает свой путь в точке 0 и за каждую минуту смещается влево или вправо на один сантиметр. Однако влево идти улитке не очень хочется, поэтому вправо она идет с вероятностью $p>1/2$, влево --- с вероятностью $(1-p)$. Справа от улитки на расстоянии $a$ находится виноградное дерево, слева на расстоянии $b$ --- шелковица.

Какова вероятность того, что улитка доползет до виноградного дерева раньше? Сколько в среднем времени ей потребуется чтобы доползти до любого из деревьев?

Решение с мартингалами: $X_{t}$ --- координата улитки в момент времени $T$ --- не мартингал, т.к. в улитка склонна смещаться вправо.

Как сварить из немартингала мартингал? Один из стандартных приемов следующий: вместо процесса $X_{t}$ рассмотреть процесс $M_{t}=u^{X_{t}}$, где число $u$ подбирается так, чтобы $M_{t}$ стал мартингалом. Из нескольких возможных $u$ выбирают $u\neq 1$, т.к при $u=1$ процесс $M_{t}$ всегда будет равен 1.

Ищем $u$ в нашем случае. Требуем, чтобы $M_{t}$ был мартингалом: $\E(M_{t+1}|\mathcal{F}_{t})=M_{t}$. Это условие упрощается до $\E(u^{\Delta{X}_{t+1}})=1$ или $p\cdot u^{1}+(1-p)\cdot u^{-1}=1$.
Квадратное уравнение имеет два корня, $u=1$ и $u=\frac{1-p}{p}$, выбираем второй.

Как и в случае симметричного случайного блуждания улитки $\P(T=\infty)=0$, $\E(T)<\infty$ так как время $T$ снова мажорируется геометрической случайной величиной. Также и процесс $M_{t\wedge T}$ ограничен сверху.

Применяя пункт (ii) теоремы Дуба получаем, что $\E(M_{T})=\E(M_{1})$. Величина $M_{T}$ принимает всего два значения: $u^{a}$ и $u^{-b}$, а $\E(M_{1})=1$. Пусть $p$ --- вероятность доползти сначала до точки $a$ (виноградного дерева). Из уравнения $pu^{a}+(1-p)u^{-b}=1$ находим $p=\frac{}{}$.

Чтобы найти $\E(T)$ можно использовать мартингал $Y_{t}=X_{t}-t\cdot \E(\Delta X_{t})$ или, что в принципе то же самое, тождество Вольда. По тождеству Вольда среднее количество слагаемых равно среднему значению суммы делить на средний размер слагаемого: $\E(T)=\frac{\E(Y_{T})}{\E(\Delta X_{t})}=...$ 


Задача. ABRACADABRA \cite{ross:scp}\cite{williams:pwm} 
% Ross, Second course, Williams, probability with martingales

Если одна мартышка печатает наугад буквы на клавиатуре, то она рано или поздно с единичной вероятностью напечатает роман Льва Толстого <<Война и мир>>. Мы поставим вопрос по-другому: сколько нажатий на клавиши в среднем потребуется чтобы напечатать слово <<абракадабра>>? Какова дисперсия требуемого количества нажатий?

Решение через мартингалы.  Организуем казино! Перед каждым нажатием в казино приходит
новый игрок с начальным капиталом в 1 рубль. Каждый входящий игрок
действует по одной и той же схеме: ставит все имеющиеся деньги на
очередную букву слова АБРАКАДАБРА (войдя в казино игрок ставит на А, потом на Б, потом на Р и т.д. Если слово кончилось, то игрок покидает казино с выигрышем). Если обезьяна напечатала нужную букву, то игрок получает свою ставку, увеличенную в 33 раза, если нет --- то игрок покидает казино без денег. 

Пусть $X_{t}$ --- суммарное благосостояние всех игроков, начавших игру до нажатия номер $t$. Величина $X_{t}$ естественно раскладывается на благосостояния отдельных игроков пришедших в казино $X_{t}=Y_{1t}+...+Y_{tt}$. Здесь $Y_{it}$ --- это благосостояние $i$-го игрока после нажатия номер $t$. Заметим, что для каждого игрока $Y_{it}$ --- это мартингал, т.к. игра справедливая --- на каждом шаге игрок либо теряет деньги, либо с вероятностью $1/33$ увеличивает свой выигрыш в 33 раза. Так как игроки начинают с одного рубля, то $\E(Y_{it})=1$. Чистый выигрыш игроков к моменту времени $t$, $M_{t}=X_{t}-t$ оказывается мартингалом: $\E(M_{t+1}-M_{t}|\mathcal{F}_{t})=\E(Y_{1,t+1}-Y_{1,t}|\mathcal{F}_{t})+...+\E(Y_{t,t+1}-Y_{t,t}|\mathcal{F}_{t})+\E(Y_{t+1,t+1}-1|\mathcal{F}_{t})=0$.

Заметим, что когда слово <<абракадабра>> напечатано впервые, все игроки кроме трех проиграли по одному рублю. Угадавший <<абракадабра>> имеет $33^{11}$, угадавший <<абра>> имеет $33^{4}$ и угадавший <<а>> имеет $33$ рубля. Значит $M_{T}=33^{11}+33^{4}+33-T$. Также получаем, что $M_{t}1_{T>t}\leq (33^{11}+33^{4}+33)1_{T>t}$.

Время $T$ можно смажорировать с помощью геометрической случайной величины. Можно давать обезьяне чистый лист после каждый 11 нажатий и засчитывать <<абракадабру>>, только если слово напечатано на отдельном листе. Очевидно, что ожидаемое время при этом возрастет. Количество листов $L$ будет иметь геометрическое распределение с $p=\frac{1}{33^{11}}$ и $\E(L)=33^{11}$. Значит $T<11L$ и $\E(T)<\E(11\cdot L)=11\cdot 33^{11}$. 

Для геометрического распределения $\P(L>t)$ стремится к нулю при $t\to\infty$. Значит $\E(M_{t}1_{T>t}\leq (33^{11}+33^{4}+33)\E(1_{T>t})\leq (33^{11}+33^{4}+33) \P(L<t/11)$ стремится к нулю.

Применяя пункт (iv) теоремы Дуба получаем, что $\E(M_{T})=\E(M_{1})$. При этом $\E(M_{T}=(33^{11}+33^{4}+33)-\E(T)$, а $\E(M_{1})=0$, значит $\E(T)=33^{11}+33^{4}+33$.

Про дисперсию. Теперь игроки приходят в наше казино с возрастающими суммами денег: первый с одним рублем, второй --- с двумя и т.д. Снова рассмотрим $X_{t}$, суммарное благосостояние игроков вступивших в игру к моменту времени $t$ и чистый выигрыш этих игроков, $M_{t}=X_{t}-(1+2+...+t)=X_{t}-\frac{t(t+1)}{2}$.

При данном изменении выигравших будет также трое и на момент появляения <<абракадабры>> общий чистый выигрыш составит $M_{T}=(T-10)\cdot 33^{11}+(T-3)\cdot 33^{4}+T\cdot 33-\frac{T(T+1)}{2}$.

Как и раньше $\E(T)<\infty$ и $\E(M_{t}1_{T>t})\to 0$ ...

Применяя пункт (iv) теоремы Дуба получаем, что $\E(M_{T})=\E(M_{1})$. Значит $\E(M_{T})=0$. Из этого уравнения выражается $\E(T^{2})$, а затем и по формуле $\Var(T)=\E(T^{2})-\E(T)^{2}$ находим дисперсию $\Var(T)=(33^{11}+33^{4}+33)^{2}-(11\cdot 33^{11}+4\cdot 33^{4}+ 33)$.


Задача. <<Следующая карта --- дама>> \cite{morters:m}, \cite{winkler:gpdp}

Перед гадалкой колода из 36 карт, хорошо перемешанная. Гадалка
должна предсказать появление дамы. Гадалка открывает
одну за другой карты из колоды и в любой момент может остановиться
и сказать <<Следующая карта будет дамой>>. Если это окажется правдой, то гадалка выиграла. 

Какую вероятность выигрыша дает следующая стратегия: дождаться появления первой дамы и после этого рискнуть и заявить, что следующая карта будет дамой? Какова оптимальная стратегия? Каковы при шансы выиграть при оптимальной стратегии?

Решение через мартингалы. Рассмотрим процесс $X_{t}$ --- долю дам в еще неоткрытой части колоды после $t$ открытых карт. После $t$ открытых карт в колоде остается $(36-t)$ карт, из которых $(36-t)X_{t}$ дам. С вероятностью $X_{t}$ (доля дам совпадает с вероятностью открыть даму) количество дам уменьшится на одну.

Считаем $\E(X_{t+1}|\mathcal{F}_{t})=\E(X_{t+1}|X_{t})=\frac{(36-t)X_{t}+X_{t}(-1)}{36-t-1}=X_{t}$.

Значит, $X_{t}$ --- мартингал. Момент остановки $T$ в любом случае ограничен 36 картами, следовательно, требования пунтка (i) теоремы Дуба выполнены и $\E(X_{T})=\E(X_{0})=\frac{4}{36}$ для любой стратегии.

Решение без мартингалов. Шансы гадалки не меняются, если она будет угадывать последнюю карту, а не следующую: информации о последней карте ровно столько, сколько о следующей. А шансы того, что последняя карта будет дамой равны $\frac{4}{36}$.
Значит все стратегии оптимальны и дают выигрыш в $4/36$

Задача. <<И в воздух чепчики бросали...>> % Ross, Second Course in Probability

Приезжающих из армии или от двора встречают $n$ женщин. Они
одновременно подбрасывают вверх $n$ чепчиков. Ловят чепчики
наугад, каждая женщина ловит один чепчик.
Женщины, поймавшие свой чепчик уходят. А женщины,
поймавшие чужой чепчик, снова подбрасывают его вверх.
Подбрасывание чепчиков продолжается до тех пор, пока каждая не
поймает свой чепчик. 

Найдите:

а) среднее количество женщин, поймавших свой чепчик при одном подбрасывании

б) среднее количество подбрасываний 

Решение с мартингалами. Пусть $N_{1}$ --- количество женщин поймавших чепчики при первом подбрасывании. Величина $N$ легко раскладывается в сумму индикаторов: $N_{1}=X_{1}+...+X_{n}$, где $X_{i}$ --- принимает значение 0 или 1 в зависимости от того, поймала ли $i$-ая женщина свой чепчик. Находим $\E(X_{i})=\P(X_{i}=1)=1/n$ и $\E(N_{1})=n\cdot \frac{1}{n}=1$. Аналогично и для остальных раундов, где участвует меньшее количество женщин.

Значит в каждом раунде в среднем одна женщина ловит свой чепчик. Пусть $Y_{t}=N_{1}+...+N_{t}$ --- суммарное количество женщин поймавших свой чепчик после раунда $t$. Получаем, что $\E(Y_{t})=t$. Случайный процесс $M_{t}=Y_{t}-t$ --- это мартингал, так как $\E(M_{t+1}|\mathcal{F}_{t})=\E(M_{t+1}|M_{t})=\E(M_{t}+N_{t+1}-1|M_{t})=M_{t}$. Разность $M_{t+1}-M_{t}=N_{t+1}-1$ ограничена количеством женщин $n$.

Смажорируем $T$, чтобы увидеть, что $\E(T)<\infty$. Вероятность того, что конкретная женщина поймает свой чепчик равна единице делить на количество женщин, участвующих в раунде, а следовательно не меньше $1/n$. Если вместо подбрасывания чепчиков в каждом раунде одна (любая) женщина будет уходить с вероятностью $1/n$, а с вероятностью $1-1/n$ не будет уходить никто, то ожидаемое количество раундов возрастет (занижена вероятность ухода, возможность ухода нескольких женщин за один раунд исключена). Количество раундов для ухода одной женщины в этом случае будет иметь геометрическое распределение со средним значением $n$ раундов, а ожидаемое общее количество раундов равно $n^{2}$. Следовательно, $\E(T)<n^{2}$.

Применяя пункт (iii) теоремы Дуба получаем, что $\E(M_{T})=\E(M_{1})$. Так как $M_{T}=Y_{T}-T=n-T$ и $\E(M_{1})=\E(N_{1}-1)=0$, получаем, что $\E(T)=n$.

Ключи и сейфы \cite{aops:keys} % aops:keys

Ballot problem \cite{ross:scp} % Ross, Second




Задача. РРО против ОРО\cite{li:ma}. % Li, Martingale Approach

Правильную монетку подбрасывают до появления последовательности РРО или ОРО. Сколько подбрасываний в среднем нужно? Какова вероятность того, что подбрасывания закончатся последовательностью РРО?

Решение с мартингалами. ???

Решение без мартингалов. ???



Задача. <<Вампиры-гладиаторы>> \cite{winkler:gpdp} % Winkler, Games people don't play

Две команды вампиров-гладиаторов борются за победу в турнире. В вашей команде 100 вампиров с силами от 1,2,..., 100. В команде противника 59 вампиров с силами 72,73,..., 130. Турнир состоит из последовательных раундов в каждом из которых участвует по одному гладиатору с каждой стороны. Если встретились гладиаторы с силами $a$ и $b$, то первый побеждает с вероятностью $\frac{a}{a+b}$, а второй --- с вероятностью $\frac{b}{a+b}$. Победитель добавляет к свой силе силу побежденного (получает силу $a+b$), а побежденный выбывает из турнира (получает силу $0$). Турнир продолжается до полного выбывания одной из команд.

Вы знаете, что команда противника будет выставлять гладиаторов по следующему принципу: на арену всегда выходит самый слабый из команды.

Какова ваша оптимальная стратегия? Какова вероятность выигрыша при этой стратегии?

Решение с мартингалами. Пусть $X_{t}$ --- суммарная сила гладиаторов нашей команды после $t$ боев. Процесс $X_{t}$ --- мартингал: $\E(X_{t+1}-X_{t}|\mathcal{F}_{t})=\frac{a}{a+b}b+\frac{b}{a+b}(-a)=0$.

Момент окончания турнира $T$ ограничен, т.к. за каждый раунд выбывает ровно один гладиатор, то есть $T<100+59$.

Применяя пункт (i) теоремы Дуба получаем, что $\E(X_{T})=\E(X_{0})$. Начальная суммарная сила нашей команды, $X_{0}=50\cdot 101$. В конце турнира суммарная сила $X_{T}$ может принимать два значения: либо 0, если мы проиграли, либо $50\cdot 101+59\cdot 101$. Если $p$ --- вероятность нашей победы, то $(1-p)\cdot 0+p\cdot (109\cdot 101)=50\cdot 101$. Значит $p=\frac{50}{109}$. Причем результат не зависит от используемой стратегии, т.к. она нигде не использовалась при подсчете!




% Задача. <<Гладиаторы>> --- а может у нее нет мартингального решения?

% Задача. Какую долю от имеющейся ставить на цвет следующей карты? 
% есть ли мартингальное решение




% Задача о втором тузе. \cite{morters:m} --- совмещена с гадалкой
% Second heart problem ---  Morters, Martingales, p. 31

% На столе хорошо перемешанная колода из 36 карт. Карты открывают одну за одной до появления первого туза. Какова вероятность того, что следующая карта будет тузом?

% Решение через мартингалы. Рассмотрим процесс $X_{t}$ --- долю тузов в еще неоткрытой % части колоды после $t$ открытых карт. После $t$ открытых карт в колоде остается % $(36-t)$ карт, из которых $(36-t)X_{t}$ тузов. С вероятностью $X_{t}$ (доля тузов % совпадает с вероятностью открыть туза) количество тузов уменьшится на один.

% Считаем %$\E(X_{t+1}|\mathcal{F}_{t})=\E(X_{t+1}|X_{t})=\frac{(36-t)X_{t}+X_{t}(-1)}{36-t-1}=X_{t}% $ .

% Значит, $X_{t}$ --- мартингал. Оптимальный момент остановки $T$ в любом случае %ограничен 36 картами, следовательно, требования пунтка (i) теоремы Дуба выполнены и $ %\E(X_{T})=\E(X_{0})=\frac{4}{36}$.

% Решение без мартингалов. Шансы открыть туза не меняются, если открывать последнюю  карту, а не следующую: информации о последней карте ровно столько, сколько о следующей. % А шансы того, что последняя карта будет тузом равны $\frac{4}{36}$.






\section{Задачи} 

Ряд задач взят из \cite{stirzaker:prp}, \cite{stirzaker:otep}, \cite{zastawniak:bsp}, \cite{blom:pspt}


Задача. 

%Усталость улитки.
%Усталость улитки в момент времени $t$ определяется как $U_{t}=t\cdot S_{t}$ (чем %дальше улитка
% эх плохо --- разный знак у S_{t} --- как бы это обозвать?

Пусть $X_{t}$ --- симметричное случайное блуждание.

Найдите $\E(TS_{T})$,

Задача. \cite{wilmott:chap} % wilmott:chap

Бабушка изготовила кисель. В банке киселя плавают 10 вишенок. За один день Вовочка выпивает случайное количество киселя равномерно распределенное от нуля до всей банки. Вовочка пьет кисель прямо с вишенками, если они ему попадаются. Чтобы бабушка ничего не заметила каждый день Вовочка доливает в банку воды до полного объема.
Вовочка пьет до тех пор пока в банке не останется 5 вишенок. Пусть $T$ --- количество дней, которые Вовочка будет пить кисель, а $X_{t}$ концентрация киселя в день $t$.

Докажите, что $\E(T)=\E(log X_{T})$ 






Задача. Вариация на тему дней рождения.

Мы набираем людей по одному в группу до тех пор, пока в группе не будет хотя бы одного совпадающего дня рождения. Пусть $T$ количество людей, которое потребуется набрать. Найдите\footnote{величина $\E(T)$ является <<некрасивой>> в том смысле, что не целая и точное значение имеет громоздкую запись. Оказывается, что $\E(T)\approx 23$, но для в решении данной задачи это не используется} $\E(T^{2})-\E(T)$. 

Решение через мартингалы. Каждый вступающий человек приходит в группу с количеством денег равным количеству людей уже вступивших в группу. Обяжем каждого вступающего человека сыграть с каждым уже вступившим в группу в такую лотерею: входящий ставит на кон 1 рубль, если их дни рождения совпадают, то входящий получает 365 рублей, если нет, то входящий теряет рубль (деньги платит и получает устроитель, существующие члены группы ничего не платят и не получают). Пусть $X_{t}$ --- чистый выигрыш всех участников после вступления в группу $t$ человек. Поскольку каждая лотерея по отдельности справедлива, то $X_{t}$ --- мартингал. 

Момент остановки $T$ ограничен сверху, $T\leq 365$. Применяя пункт (i) теоремы Дуба получаем, что $\E(X_{T})=\E(X_{1})$. При этом $\E(X_{1})=0$, а $X_{T}=365-(1+2+...+(T-1))=365-\frac{T(T-1)}{2}$. И $\E(T^{2}-T)=2\cdot 365$.

Задача. Улитка отдыхает (симметричное блуждание), \cite{blom:pspt}
% r-in-advance game, Blom, Problems and Spanshots from probability theory

Улитка начинает свой путь в точке 0 и за каждую минуту равновероятно смещается влево или вправо на один сантиметр или отдыхает никуда не перемещаясь. Влево (и вправо) улитка ползет с одинаковой вероятностью $p$, отдыхает --- с вероятностью $(1-2p)$. 

Справа от улитки на расстоянии $a$ находится виноградное дерево, слева на расстоянии $b$ --- шелковица.

Какова вероятность того, что улитка доползет до виноградного дерева раньше? Сколько в среднем времени ей потребуется чтобы доползти до любого из деревьев?



Задача. Улитка отдыхает (несимметричное блуждание), \cite{blom:pspt}
% r-in-advance game, Blom, Problems and Spanshots from probability theory

Улитка начинает свой путь в точке 0 и за каждую минуту смещается влево или вправо на один сантиметр или отдыхает никуда не перемещаясь. Влево улитка ползет с вероятностью $p_{l}$, вправо --- с вероятностью $p_{r}$, отдыхает --- с вероятностью $(1-p_{l}-p_{r})$. 

Справа от улитки на расстоянии $a$ находится виноградное дерево, слева на расстоянии $b$ --- шелковица.

Какова вероятность того, что улитка доползет до виноградного дерева раньше? Сколько в среднем времени ей потребуется чтобы доползти до любого из деревьев?

Задача. Ждем 1 рубль. 

Найдите $\E(T)$


Задача. Войны Добра и Зла \cite{stirzaker:otep}, 12-13

Изначально воюют один воин Добра и один воин Зла. Каждый день судьба выбирает одного из воюющих наугад и добавляет еще одного воина той же стороны. Никто никогда не погибает, они просто сражаются. Пусть $T$ --- время, когда Судьба впервые добавит война Добра.  Найдите $\E(1/(T+2)) $.

Пусть $ X_{t} $ --- доля войнов Добра в конце дня $ t $.



Задача. Банковский счет улитки. \cite{stirzaker:prp}, 12.5

Перед отправкой в свой долгий путь улитка положила все свои сбережения (один рубль) в банк. За каждую минуту банк начисляет небольшой процент $r>0$, такой что $0<r<1/cos(\frac{\pi}{a+b})-1$. Какая сумма в среднем будет находится на счету улитки к моменту достижения ей любого из деревьев?

Задача. Сумма координат улитки. \cite{stirzaker:prp}, 12.7

Каждый раз проходя через точку с координатой $x$ улитка получает $x$ рублей. Каков суммарный заработок улитки?

% Какова средняя координата? --- считается ли?

Задача. Улитка на плоскости.  \cite{stirzaker:otep}, 12-17





% Задача. Улитка на склоне. (???) --- уже скучновато про улитку



Задача. Есть неправильная монетка, орел выпадает с вероятностью $p$. Монетку подкидывают до тех пор, пока впервые не выпадет два орла подряд. 
\begin{enumerate}
\item Сколько в среднем потребуется подбрасываний?
\item Сколько в среднем окажется орлов?
\end{enumerate}

Решение. Пусть $T$ -- момент остановки, когда впервые выпадает два орла, а $N_{t}$ -- количество орлов в момент $t$. $\E(T)$ находится через марковские цепи, системой линейных уравнений. Пусть $T_{1}$ -- сколько нужно ждать до двух орлов подряд, если мы только что выпал один орел:
\begin{equation}
\begin{cases}
\E(T)=1+p\E(T_{1})+(1-p)\E(T)\\
\E(T_{1})=1+p\cdot 0+(1-p)\E(T)
\end{cases}
\end{equation}
Замечаем, что $M_{t}=N_{t}-pt$ -- мартингал. Условия теоремы Дуба выполнены, поэтому $\E(N_{T})=p\E(T)$.



%\bibliography{e:/documents/tex_general/opit} 
% название файла с коллекцией названий статей/книг

% источники:
% Ross, Second course in probability
% Stirzaker, Probability and random processes
% Stirzaker, One thousand exercises in probability
% Williams, Probability with martingales
% Morters, Martingales
% Chang, Stochastic processes
% Blom, Problems and snapshots from probability theory
% Li, Martingale Approach to the Study of Occurrence of Sequence Patterns in  Repeated % Experiments
% Winkler, Games people don't play
% Zastawniak, Basic stochastic processes
% упр. 3.12 $(-1)^{\tau}=(-1)^{K}$ без всяких ожиданий!

% Shreve, Stochastic calculus for finance I --- ??? (пока не включен, а что там было?)
% aops, keys
% wilmott, ts_t




%\part{Метод моментов}
%\input{../meth_moments/meth_moments.tex}

%\part{Проверка гипотез}
%\input{../test_hypo/test_hypo.tex}

%\part{Трюк с окружностью}
%\input{../circle_trick/circle_trick.tex}





\bibliography{/home/boris/science/tex_general/opit}


\printindex % печать предметного указателя здесь


\end{document}
