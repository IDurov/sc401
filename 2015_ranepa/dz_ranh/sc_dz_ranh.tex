\documentclass[10pt,a4paper]{article}
\usepackage[utf8]{inputenc}
\usepackage[russian]{babel}

\usepackage{amsmath}
\usepackage{amsfonts}
\usepackage{amssymb}
\usepackage[left=2cm,right=2cm,top=2cm,bottom=2cm]{geometry}
\begin{document}

\begin{enumerate}
\item Найдите мощность множества всех решений уравнения $x_1+x_2+x_3=0$ в целых числах. 
\item Выпишите минимальную $\sigma$-алгебру, порождаемую множествами $A=[-2;5)$ и $B=[-2;0]$ на числовой прямой.
\item Совместный закон распределения $X$ и $Y$ задан табличкой

\begin{tabular}{c|cc}
 & $Y=2$ & $Y=5$ \\ 
\hline 
$X=0$ & $0.1$ & $0.3$ \\ 
$X=1$ & $0.3$ & $0.3$ \\ 
\end{tabular} 

Найдите $E(Y|X)$, $Var(Y|X)$, $Var(E(Y|X))$

Выпишите все события из  $\sigma$-алгебры $\mathcal{F}=\sigma(X\cdot Y)$

\item Пусть $X$ и $Y$ --- независимые случайные величины, равные 1 с
вероятностью $0.2$ или 0 с вероятностью $0.8$. Пусть $Z=1_{X+Y=0}$. Найдите
$E(X|Z)$. 

\item Известно, что $E(Y|X)=0$. Может ли быть отличной от нуля величина $E(Y)$? $Cov(Y,X)$? $Cov(Y^2,X)$? $Cov(Y, X^2)$?

\item Пусть $S_n$ --- симметричное случайное блуждание. Верно ли, что мартингалом является $S_n^2/n$? $S_n^3-3nS_n$?

\item Саша и Маша играют в шахматы много партий подряд. За выигрыш победитель получает одно очко, проигравший --- ноль. За ничью оба получают по половине очка. Маша выигрывает с вероятностью $0.4$, Саша --- с вероятностью $0.3$. Обозначим $X_t$ --- разницу набранных очков (Маша минус Саша) в момент времени $t$. Они заканчивают играть в момент времени $\tau$, когда разница набранных очков достигнет двух.

Какие значения потенциально принимает $X_{\tau}$?

Является ли процесс $X_t$ мартингалом?

При каком $a$ процесс $M_t=a^{X_t}$ будет мартингалом?

При каком $b$ процесс $L_t=X_t - bt$ будет мартингалом?

Найдите вероятность того, что в результате по очкам выиграет Саша. Мартингал $M_t$ вам в помощь.

Найдите среднюю продолжительность партии. Мартингал $L_t$ вам в помощь.




\end{enumerate}


\end{document}