\documentclass[pdftex,12pt,a4paper]{article}

\input{/home/boris/Dropbox/Public/tex_general/title_bor_utf8}


%\input{/home/boris/Dropbox/Public/tex_general/prob_and_sol_utf8}

%\title{Задачи по элементарной теории вероятностей и матстатистике}
%\author{Составитель: Борис Демешев, boris.demeshev@gmail.com}
%\date{\today}

\begin{document}

\begin{enumerate}
\item Распределение случайных величин $X$ и $Y$ задано вероятностями $\P(X=i,Y=j)=0{,}1$ при $1\leq i \leq j \leq 4$. Найдите $\E(Y\mid X)$. 
\item Величина $X$ равномерна на отрезке $[0;1]$, величина $Y$ равновероятно принимает значения $0$ и $1$. Величины $X$ и $Y$ независимы, $Z=X^Y$. Найдите $\E(Z\mid Y)$, $Var(Z\mid Y)$, $\E(Z\mid X)$, $Var(Z\mid X)$.
\item Величины $X_1$, ..., $X_n$ независимы и равномерна на отрезке $[0;1]$. Найдите $\E(X_1\mid \min\{X_1,...,X_n\})$ и $\E(X_1\mid \max\{X_1,...,X_n\})$.
\item Величины $X_1$, ..., $X_n$ независимы и одинаково распределены, $S_n=X_1+...+X_n$. Найдите $\E(S_k \mid S_n )$ в двух случаях: $k\leq n$ и $k>n$.
\item Величина $X$ равномерна на отрезке $[0;1]$. В шляпе лежат две свернутые бумажки. На одной бумажке написано $X$, на другой $X^{2}$. Вы тяните одну бумажку наугад. Пусть $Z$ --- число, написанное на вытянутой Вами бумажке, а $W$ - число на другой бумажке. Увидев число Вы решаете, оставить себе эту бумажку, или отказаться от этой и забрать оставшуюся. Ваш выигрыш - число на оставшейся у Вас бумажке. 
\begin{enumerate}
\item Найдите $\E(W|Z)$ 
\item Максимально подробно (кубическое уравнение там будет суровое, не решайте его) опишите стратегию максимизирующую Ваш выигрыш 
\end{enumerate}

\end{enumerate}

%\pagestyle{myheadings} \markboth{ТВИМС-задачник. Демешев Борис. roah@yandex.ru }{ТВИМС-задачник. Демешев Борис. roah@yandex.ru }
%\maketitle
%\tableofcontents{}

%\parindent=0 pt % отступ равен 0



\bibliography{/home/boris/Dropbox/Public/tex_general/opit}
\printindex % печать предметного указателя здесь

\end{document}