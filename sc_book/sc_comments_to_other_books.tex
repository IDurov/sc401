% это может пригодится...

К финалу семинара 3: Про $y=2x+3$... \\

Теперь перейдем к конкретной $f(x)=2x+3$ \\
Упр. Докажите, что для $f(x)=2x+3$ прообраз любого множества вида $[a;b)$ будет хорошим (борелевским). \\
Мы доказывали, что если $\mathcal{A}$ - набор множеств вида $[a;b)$, то $\sigma (A)=\mathcal{B}$ (множества вида $[a;b)$ порождают борелевскую $\sigma$-алгебру). \\
Собираем паззл: \\
1. $[a;b)\in\mathcal{G}$, т.е. $\mathcal{G}$ - это $\sigma$-алгебра, содержащая все $[a;b)$ \\
2. $\mathcal{B}=\sigma (\mathcal{A})$, т.е. $\mathcal{B}$ - это минимальная $\sigma$-алгебра, содержащая все $[a;b)$ \\
Значит $\mathcal{B}\subseteq \mathcal{G}$. Т.е. прообраз любого борелевского множества борелевский. \\

Упр. Докажите, что $y(x)=cos(x)$ - борелевская функция \\




% это всякая фигня, вырезанная из sk_skelet
Комменты к популярным книжкам \\
Очепятки... \\
$[$Aad$]$: \\
Cтр. 5: \\
3 снизу: видимо вместо four должно быть two: \\
А именно, требуется применить неравенство 1.19 к:
$E|X|1_{|X_{n}|>M}$ и $E|X_{n}|1_{|X|>M}$ \\
Стр. 10: \\
1 сверху: вместо Theorem 1.18 должно быть Lemma 1.21 \\
Стр. 15: \\
3 снизу: вместо $X 1_{F}$ должно быть $X_{n} 1_{F}$ \\
2 снизу: вместо $E(X_{n}|\mathcal{F}_{m})$ должно быть
$E(X_{\infty}|\mathcal{F}_{m})$ \\

$[$Steele$]$: \\
p. 21: \\
4 строка снизу: должно быть $||X_{n}||_{p}^{p}=E(X_{n}^{p})=...$ \\
p. 22: \\
Должно быть $\epsilon<\frac{b-a}{2}$ \\
p. 24: \\
3 строка сверху: должно быть $S\subset
\{\tau<\infty\}\subset\{...\}$ \\
Тот  же случай. \\
p. 26: \\
В определении $A_{ab}$ исправить нестрогие неравенства на строгие. \\

Comments... \\
$[$Steele$]$: \\
p. 21: \\
Для $\forall\lambda\exists N$, что при $n>N$ события $\{X\wedge
n\ge\lambda\}=\{X\ge \lambda\}$. Именно поэтому переход от $X$ к
$X\wedge n$ сохраняет применимость неравенства Дуба. \\
p. 22: \\
Может ли RHS равняться $+\infty$? Скорее да. \\
p. 22-23: \\
По-моему, следует заменить $\sup_{m\le k<\infty}|M_{k}-M_{m}|\ge
\epsilon$ на $\sup_{m\le k<\infty}|M_{k}-M_{m}|> \epsilon$. \\
Почему? \\
Неравенство Дуба применимо для $\sup_{0\le k\le n}M_{k}{'}$, а
применять его требуется к $\sup_{0\le k<\infty}M_{k}{'}$. \\
Для строго неравенства возможно такое разложение: \\
$\sup_{0\le
k<\infty}M_{k}{'}>\epsilon^{2}=\bigcup_{n}\left(\sup_{0\le k\le
n}M_{k}{'}>\epsilon^{2}\right)$. \\
При аккуратном доказательстве потребуется ссылка на непрерывность
вероятности. \\


Steele solutions: \\
2.1. a. \\
$3x^{3}+45x^{2}-100x+52=0$ \\
$x_{1}=1$, $x_{2}=\frac{-24+\sqrt{24^{2}+3\cdot 52}}{3}$,
$x_{2}=\frac{-24-\sqrt{24^{2}+3\cdot 52}}{3}$ \\
2.1. b. ?? \\
$1=c_{1}+c_{2}x_{2}^{101}+c_{3}x_{3}^{101}$ \\
$1=c_{1}+c_{2}x_{2}^{100}+c_{3}x_{3}^{100}$ \\
$0=c_{1}+c_{2}x_{2}^{-100}+c_{3}x_{3}^{-100}$ \\
$c_{1}+c_{2}+c_{3}$ - ? \\
2.5. \\
$M_{n}$ - субмартингал $\Rightarrow$ $M_{n\wedge \tau}$ -
субмартингал. \\
$M_{n\wedge \nu}=M_{n\wedge\tau\wedge\nu}$, следовательно,
$M_{n\wedge \nu}$ является мартингальным преобразованием для
$M_{n\wedge\tau}$ \\
Применяем proposition 2.1. (p. 24), $E(M_{n\wedge \nu})\le
E(M_{n\wedge \tau})$. \\
Выбираем $n$ большее, чем граница для $\tau$ и $\nu$ \\


3.5. \\
3.17 left? правая: домножаем подынтегральное выражение на
$\frac{t}{x}$ (интеграл по $dt$) \\
 $t \in [0;1] \Rightarrow e^{-\frac{t^{2}}{2}} \in
[e^{-\frac{1}{2}};1]$ (заменяем $e^{-\frac{t^{2}}{2}}$ на константу) \\

