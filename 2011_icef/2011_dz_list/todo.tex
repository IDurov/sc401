\documentclass[pdftex,12pt,a4paper]{article}

% задаем более широкие поля
\emergencystretch=2em \voffset=-2cm \hoffset=-1cm
\unitlength=0.6mm \textwidth=17cm \textheight=25cm



\usepackage{makeidx}



\usepackage{cmap}

\usepackage[pdftex]{graphicx} % omit pdftex option if not using pdflatex

\usepackage[colorlinks,hyperindex,unicode]{hyperref}


\usepackage[utf8]{inputenc}
\usepackage[T2A]{fontenc} 
\usepackage[russian]{babel}

\usepackage{amssymb}
\usepackage{amsmath}
\usepackage{amsthm}
\usepackage{epsfig}
\usepackage{bm}
\usepackage{color}

\usepackage{multicol}


%моя добавка
\usepackage{ifpdf}
\ifpdf
	\DeclareGraphicsRule{*}{mps}{*}{}
\fi
% конец добавки

\usepackage{asymptote} % After graphicx!
\usepackage{sagetex} % i suppose after graphicx also...


\title{Это документ!}


\begin{document}
%\maketitle
Что порешать долгими зимними вечерами по стохастическому анализу?
\begin{enumerate}
\item Упражнения в конце лекций Зервоса. Смотрите приложенный прошлогодний вариант.
\item Два промежуточных теста прошлого года.
\item Из бета-версии задачника:
\begin{enumerate}
\item 2: 32-37
\item 8: все, что душе угодно
\item раздел 10.2: все, что хочется
\item 11: 1, 2, 6-11, 14, 30, 32, 35
\item 12: все, что нравится
\item 13.1, 14.1-14.3
\end{enumerate}
\end{enumerate}


\parindent=0 pt % no indent

\end{document}
