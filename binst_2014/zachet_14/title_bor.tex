%%%%%%%%%%%%%%%%%%%%%%%  Загрузка пакетов  %%%%%%%%%%%%%%%%%%%%%%%%%%%%%%%%%%

%\usepackage{showkeys} % показывать метки в готовом pdf 

\usepackage{etex} % расширение классического tex
% в частности позволяет подгружать гораздо больше пакетов, чем мы и займёмся далее

%\usepackage{mathtext} % русские буквы в формулах? (и без неё работает)
% Например, $x_{\text{один}}$

%\usepackage{cmap} % для поиска русских слов в pdf --- теперь устарело?
\usepackage{verbatim} % для многострочных комментариев
\usepackage{makeidx} % для создания предметных указателей
\usepackage[X2,T2A]{fontenc}
\usepackage[utf8]{inputenc} % задание utf8 кодировки исходного tex файла
\usepackage{setspace}
\usepackage{amsmath,amsfonts,amssymb,amsthm}
\usepackage{mathrsfs} % sudo yum install texlive-rsfs
\usepackage{dsfont} % sudo yum install texlive-doublestroke
\usepackage{array,multicol,multirow,bigstrut} % sudo yum install texlive-multirow
\usepackage{indentfirst} % установка отступа в первом абзаце главы
\usepackage[british,russian]{babel} % выбор языка для документа
\usepackage{bm}
\usepackage{bbm} % шрифт с двойными буквами
\usepackage[perpage]{footmisc}

\usepackage{dcolumn} % центрирование по разделителю для apsrtable

% создание гиперссылок в pdf
\usepackage[pdftex,unicode,colorlinks=true,urlcolor=blue,hyperindex,breaklinks]{hyperref} 

% свешиваем пунктуацию 
% теперь знаки пунктуации могут вылезать за правую границу текста, при этом текст выглядит ровнее
\usepackage{microtype}

\usepackage{textcomp}  % Чтобы в формулах можно было русские буквы писать через \text{}

% размер листа бумаги
\usepackage[paper=a4paper,top=20mm, bottom=13.5mm,left=30mm,right=15mm,includefoot]{geometry}

\usepackage{xcolor}

\usepackage[pdftex]{graphicx} % для вставки графики 

\usepackage{float,longtable}
\usepackage{soulutf8}

\usepackage{enumitem} % дополнительные плюшки для списков
%  например \begin{enumerate}[resume] позволяет продолжить нумерацию в новом списке

\usepackage{mathtools}
\usepackage{cancel,xspace} % sudo yum install texlive-cancel

%\usepackage{minted} % display program code with syntax highlighting
% требует установки pygments и python 

\usepackage{numprint} % sudo yum install texlive-numprint
\npthousandsep{,}\npthousandthpartsep{}\npdecimalsign{.}

\usepackage{embedfile} % Чтобы код LaTeXа включился как приложение в PDF-файл

\usepackage{subfigure} % для создания нескольких рисунков внутри одного

\usepackage{tikz,pgfplots} % язык для рисования графики из latex'a
\usetikzlibrary{trees} % tikz-прибамбас для рисовки деревьев
\usepackage{tikz-qtree} % альтернативный tikz-прибамбас для рисовки деревьев
\usetikzlibrary{arrows} % tikz-прибамбас для рисовки стрелочек подлиннее

\usepackage{todonotes} % для вставки в документ заметок о том, что осталось сделать
% \todo{Здесь надо коэффициенты исправить}
% \missingfigure{Здесь будет Последний день Помпеи}
% \listoftodos --- печатает все поставленные \todo'шки


% более красивые таблицы
\usepackage{booktabs}
% заповеди из докупентации: 
% 1. Не используйте вертикальные линни
% 2. Не используйте двойные линии
% 3. Единицы измерения - в шапку таблицы
% 4. Не сокращайте .1 вместо 0.1
% 5. Повторяющееся значение повторяйте, а не говорите "то же"



%\usepackage{asymptote} % пакет для рисовки графики, должен идти после graphics
% но мы переходим на tikz :)

%\usepackage{sagetex} % для интеграции с Sage (вероятно тоже должен идти после graphics)

% metapost создает упрощенные eps файлы, которые можно напрямую включать в pdf 
% эта группа команд декларирует, что файлы будут этого упрощенного формата
% если metapost не используется, то этот блок не нужен
\usepackage{ifpdf} % для определения, запускается ли pdflatex или просто латех
\ifpdf
	\DeclareGraphicsRule{*}{mps}{*}{}
\fi
%%%%%%%%%%%%%%%%%%%%%%%%%%%%%%%%%%%%%%%%%%%%%%%%%%%%%%%%%%%%%%%%%%%%%%


%%%%%%%%%%%%%%%%%%%%%%%  Внедрение tex исходников в pdf файл  %%%%%%%%%%%%%%%%%%%%%%%%%%%%%%%%%%
%\embedfile[desc={Main tex file}]{\jobname.tex} % Включение кода в выходной файл
%\embedfile[desc={title_bor}]{/home/boris/science/tex_general/title_bor_utf8.tex}

%%%%%%%%%%%%%%%%%%%%%%%%%%%%%%%%%%%%%%%%%%%%%%%%%%%%%%%%%%%%%%%%%%%%%%



%%%%%%%%%%%%%%%%%%%%%%%  ПАРАМЕТРЫ  %%%%%%%%%%%%%%%%%%%%%%%%%%%%%%%%%%
\setstretch{1}                          % Межстрочный интервал
\flushbottom                            % Эта команда заставляет LaTeX чуть растягивать строки, чтобы получить идеально прямоугольную страницу
\righthyphenmin=2                       % Разрешение переноса двух и более символов
\pagestyle{plain}                       % Нумерация страниц снизу по центру.
\widowpenalty=300                     % Небольшое наказание за вдовствующую строку (одна строка абзаца на этой странице, остальное --- на следующей)
\clubpenalty=3000                     % Приличное наказание за сиротствующую строку (омерзительно висящая одинокая строка в начале страницы)
\setlength{\parindent}{1.5em}           % Красная строка.
%\captiondelim{. }
\setlength{\topsep}{0pt}
%%%%%%%%%%%%%%%%%%%%%%%%%%%%%%%%%%%%%%%%%%%%%%%%%%%%%%%%%%%%%%%%%%%%%%



%%%%%%%% Это окружение, которое выравнивает по центру без отступа, как у простого center
\newenvironment{center*}{%
  \setlength\topsep{0pt}
  \setlength\parskip{0pt}
  \begin{center}
}{%
  \end{center}
}
%%%%%%%%%%%%%%%%%%%%%%%%%%%%%%%%%%%%%%%%%%%%%%%%%%%%%%%%%%%%%%%%%%%%%%


%%%%%%%%%%%%%%%%%%%%%%%%%%% Правила переноса  слов
\hyphenation{ }
%%%%%%%%%%%%%%%%%%%%%%%%%%%%%%%%%%%%%%%%%%%%%%%%%%%%%%%%%%%%%%%%%%%%%%

\emergencystretch=2em


% DEFS
\def \mbf{\mathbf}
\def \msf{\mathsf}
\def \mbb{\mathbb}
\def \tbf{\textbf}
\def \tsf{\textsf}
\def \ttt{\texttt}
\def \tbb{\textbb}

\def \wh{\widehat}
\def \wt{\widetilde}
\def \ni{\noindent}
\def \ol{\overline}
\def \cd{\cdot}
\def \fr{\frac}
\def \bs{\backslash}
\def \lims{\limits}
\DeclareMathOperator{\dist}{dist}
\DeclareMathOperator{\VC}{VCdim}
\DeclareMathOperator{\card}{card}
\DeclareMathOperator{\sign}{sign}
\DeclareMathOperator{\sgn}{sign}
\DeclareMathOperator{\Tr}{\mbf{Tr}}
\DeclareMathOperator{\tr}{tr}


\def \xfs{(x_1,\ldots,x_{n-1})}
\DeclareMathOperator*{\argmin}{arg\,min}
\DeclareMathOperator*{\amn}{arg\,min}
\DeclareMathOperator*{\amx}{arg\,max}
\DeclareMathOperator{\trace}{tr}


\DeclareMathOperator{\Corr}{Corr}
\DeclareMathOperator{\sCorr}{sCorr}
\DeclareMathOperator{\sCov}{sCov}
\DeclareMathOperator{\sVar}{sVar}

\DeclareMathOperator{\argmax}{argmax}
\DeclareMathOperator{\Cov}{Cov}
\DeclareMathOperator{\Var}{Var}
\DeclareMathOperator{\corr}{Corr}
\DeclareMathOperator{\cov}{Cov}
\DeclareMathOperator{\var}{Var}
\DeclareMathOperator{\bin}{Bin}
\DeclareMathOperator{\Bin}{Bin}
\DeclareMathOperator{\rang}{rang}
\DeclareMathOperator*{\plim}{plim}
\DeclareMathOperator{\MSE}{MSE}

\providecommand{\iff}{\Leftrightarrow}
\providecommand{\hence}{\Rightarrow}

\def \ti{\tilde}
\def \wti{\widetilde}

\def \mA{\mathcal{A}}
\def \mB{\mathcal{B}}
\def \mC{\mathcal{C}}
\def \mE{\mathcal{E}}
\def \mF{\mathcal{F}}
\def \mH{\mathcal{H}}
\def \mL{\mathcal{L}}
\def \mN{\mathcal{N}}
\def \mU{\mathcal{U}}
\def \mV{\mathcal{V}}
\def \mW{\mathcal{W}}


\def \R{\mbb R}
\def \N{\mbb N}
\def \Z{\mbb Z}
\def \P{\mbb{P}}
\def \p{\mbb{P}}
\newcommand{\E}{\mathbb{E}}
\def \D{\msf{D}}
\def \I{\mbf{I}}

\def \QQ{\mbb Q}
\def \RR{\mbb R}
\def \NN{\mbb N}
\def \ZZ{\mbb Z}
\def \PP{\mbb P}


\def \a{\alpha}
\def \b{\beta}
\def \t{\tau}
\def \dt{\delta}
\newcommand{\e}{\varepsilon}
\def \ga{\gamma}
\def \kp{\varkappa}
\def \la{\lambda}
\def \sg{\sigma}
\def \sgm{\sigma}
\def \tt{\theta}
\def \ve{\varepsilon}
\def \Dt{\Delta}
\def \La{\Lambda}
\def \Sgm{\Sigma}
\def \Sg{\Sigma}
\def \Tt{\Theta}
\def \Om{\Omega}
\def \om{\omega}

%\newcommand{\p}{\partial}

\def \ni{\noindent}
\def \lq{\glqq}
\def \rq{\grqq}
\def \lbr{\linebreak}
\def \vsi{\vspace{0.1cm}}
\def \vsii{\vspace{0.2cm}}
\def \vsiii{\vspace{0.3cm}}
\def \vsiv{\vspace{0.4cm}}
\def \vsv{\vspace{0.5cm}}
\def \vsvi{\vspace{0.6cm}}
\def \vsvii{\vspace{0.7cm}}
\def \vsviii{\vspace{0.8cm}}
\def \vsix{\vspace{0.9cm}}
\def \VSI{\vspace{1cm}}
\def \VSII{\vspace{2cm}}
\def \VSIII{\vspace{3cm}}

\newcommand{\grad}{\mathrm{grad}}
\newcommand{\bls}[1]{\boldsymbol{#1}}
\newcommand{\bsA}{\boldsymbol{A}}
\newcommand{\bsH}{\boldsymbol{H}}
\newcommand{\bsI}{\boldsymbol{I}}
\newcommand{\bsP}{\boldsymbol{P}}
\newcommand{\bsR}{\boldsymbol{R}}
\newcommand{\bsS}{\boldsymbol{S}}
\newcommand{\bsX}{\boldsymbol{X}}
\newcommand{\bsY}{\boldsymbol{Y}}
\newcommand{\bsZ}{\boldsymbol{Z}}
\newcommand{\bse}{\boldsymbol{e}}
\newcommand{\bsq}{\boldsymbol{q}}
\newcommand{\bsy}{\boldsymbol{y}}
\newcommand{\bsbeta}{\boldsymbol{\beta}}
\newcommand{\fish}{\mathrm{F}}
\newcommand{\Fish}{\mathrm{F}}
\renewcommand{\phi}{\varphi}
\newcommand{\ind}{\mathds{1}}
\newcommand{\inds}[1]{\mathds{1}_{\{#1\}}}
\renewcommand{\to}{\rightarrow}
\newcommand{\sumin}{\sum\limits_{i=1}^n}
\newcommand{\ofbr}[1]{\bigl( \{ #1 \} \bigr)}     % Например, вероятность события. Большие круглые, нормальные фигурные скобки вокруг аргумента
\newcommand{\Ofbr}[1]{\Bigl( \bigl\{ #1 \bigr\} \Bigr)} % Например, вероятность события. Больше больших круглые, большие фигурные скобки вокруг аргумента
\newcommand{\oeq}{{}\textcircled{\raisebox{-0.4pt}{{}={}}}{}} % Равно в кружке
\newcommand{\og}{\textcircled{\raisebox{-0.4pt}{>}}}  % Знак больше в кружке

% вместо горизонтальной делаем косую черточку в нестрогих неравенствах
\renewcommand{\le}{\leqslant}
\renewcommand{\ge}{\geqslant}
\renewcommand{\leq}{\leqslant}
\renewcommand{\geq}{\geqslant}


\newcommand{\figb}[1]{\bigl\{ #1  \bigr\}} % большие фигурные скобки вокруг аргумента
\newcommand{\figB}[1]{\Bigl\{ #1  \Bigr\}} % Больше больших фигурные скобки вокруг аргумента
\newcommand{\parb}[1]{\bigl( #1  \bigr)}   % большие скобки вокруг аргумента
\newcommand{\parB}[1]{\Bigl( #1  \Bigr)}   % Больше больших круглые скобки вокруг аргумента
\newcommand{\parbb}[1]{\biggl( #1  \biggr)} % большие-большие круглые скобки вокруг аргумента
\newcommand{\br}[1]{\left( #1  \right)}    % круглые скобки, подгоняемые по размеру аргумента
\newcommand{\fbr}[1]{\left\{ #1  \right\}} % фигурные скобки, подгоняемые по размеру аргумента
\newcommand{\eqdef}{\mathrel{\stackrel{\rm def}=}} % знак равно по определению
\newcommand{\const}{\mathrm{const}}        % const прямым начертанием
\newcommand{\zdt}[1]{\textit{#1}}
\newcommand{\ENG}[1]{\foreignlanguage{british}{#1}}
\newcommand{\ENGs}{\selectlanguage{british}}
\newcommand{\RUSs}{\selectlanguage{russian}}
\newcommand{\iid}{\text{i.\hspace{1pt}i.\hspace{1pt}d.}}

\newdimen\theoremskip
\theoremskip=0pt
\newenvironment{note}{\par\vskip\theoremskip\textbf{Замечание.\xspace}}{\par\vskip\theoremskip}
\newenvironment{hint}{\par\vskip\theoremskip\textbf{Подсказка.\xspace}}{\par\vskip\theoremskip}
\newenvironment{ist}{\par\vskip\theoremskip Источник:\xspace}{\par\vskip\theoremskip}

\newcommand*{\tabvrulel}[1]{\multicolumn{1}{|c}{#1}}
\newcommand*{\tabvruler}[1]{\multicolumn{1}{c|}{#1}}

\newcommand{\II}{{\fontencoding{X2}\selectfont\CYRII}}   % I десятеричное (английская i неуместна)
\newcommand{\ii}{{\fontencoding{X2}\selectfont\cyrii}}   % i десятеричное
\newcommand{\EE}{{\fontencoding{X2}\selectfont\CYRYAT}}  % ЯТЬ
\newcommand{\ee}{{\fontencoding{X2}\selectfont\cyryat}}  % ять
\newcommand{\FF}{{\fontencoding{X2}\selectfont\CYROTLD}} % ФИТА
\newcommand{\ff}{{\fontencoding{X2}\selectfont\cyrotld}} % фита
\newcommand{\YY}{{\fontencoding{X2}\selectfont\CYRIZH}}  % ИЖИЦА
\newcommand{\yy}{{\fontencoding{X2}\selectfont\cyrizh}}  % ижица

%%%%%%%%%%%%%%%%%%%%% Определение разрядки разреженного текста и задание красивых многоточий
\sodef\so{}{.15em}{1em plus1em}{.3em plus.05em minus.05em}
\newcommand{\ldotst}{\so{...}}
\newcommand{\ldotsq}{\so{?\hbox{\hspace{-0.61ex}}..}}
\newcommand{\ldotse}{\so{!..}}
%%%%%%%%%%%%%%%%%%%%%%%%%%%%%%%%%%%%%%%%%%%%%%%%%%%%%%%%%%%%%%%%%%%%%%

%%%%%%%%%%%%%%%%%%%%%%%%%%%%% Команда для переноса символов бинарных операций
\def\hm#1{#1\nobreak\discretionary{}{\hbox{$#1$}}{}}
%%%%%%%%%%%%%%%%%%%%%%%%%%%%%%%%%%%%%%%%%%%%%%%%%%%%%%%%%%%%%%%%%%%%%%

\setlist[enumerate,1]{label=\arabic*., ref=\arabic*, partopsep=0pt plus 2pt, topsep=0pt plus 1.5pt,itemsep=0pt plus .5pt,parsep=0pt plus .5pt}
\setlist[itemize,1]{partopsep=0pt plus 2pt, topsep=0pt plus 1.5pt,itemsep=0pt plus .5pt,parsep=0pt plus .5pt}

% Эти парни затем, если вдруг не захочется управлять списками из-под уютненького enumitem
% или если будет жизненно важно, чтобы в списках были именно русские буквы.
%\setlength{\partopsep}{0pt plus 3pt}
%\setlength{\topsep}{0pt plus 2pt}
%\setlength{\itemsep}{0 plus 1pt}
%\setlength{\parsep}{0 plus 1pt}

%на всякий случай пока есть
%теоремы без нумерации и имени
%\newtheorem*{theor}{Теорема}

%"Определения","Замечания"
%и "Гипотезы" не нумеруются
%\newtheorem*{defin}{Определение}
%\newtheorem*{rem}{Замечание}
%\newtheorem*{conj}{Гипотеза}

%"Теоремы" и "Леммы" нумеруются
%по главам и согласованно м/у собой
%\newtheorem{theorem}{Теорема}
%\newtheorem{lemma}[theorem]{Лемма}

% Утверждения нумеруются по главам
% независимо от Лемм и Теорем
%\newtheorem{prop}{Утверждение}
%\newtheorem{cor}{Следствие} 